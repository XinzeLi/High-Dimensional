\documentclass[12pt]{article}
\usepackage[english]{babel}
\usepackage[utf8x]{inputenc}
\usepackage[T1]{fontenc}
\usepackage{scribe}
\usepackage{listings}

\usepackage{hyperref}
\usepackage{cleveref}
%\usepackage[style=authoryear]{biblatex}
\renewcommand{\Pr}{{\bf Pr}}
\newcommand{\Prx}{\mathop{\bf Pr\/}}
\newcommand{\E}{{\bf E}}
\newcommand{\Ex}{\mathop{\bf E\/}}
\newcommand{\Var}{{\bf Var}}
\newcommand{\Varx}{\mathop{\bf Var\/}}
\newcommand{\Cov}{{\bf Cov}}
\newcommand{\Cor}{{\bf Corr}}
\newcommand{\Covx}{\mathop{\bf Cov\/}}
\newcommand{\rank}{\text{rank}}

% shortcuts for symbol names that are too long to type
\newcommand{\eps}{\epsilon}
\newcommand{\lam}{\lambda}
\renewcommand{\l}{\ell}
\newcommand{\la}{\langle}
\newcommand{\ra}{\rangle}
%\newcommand{\wh}{\widehat}

% "blackboard-fonted" letters for the reals, naturals etc.
\newcommand{\R}{\mathbb R}
\newcommand{\N}{\mathbb N}
\newcommand{\Z}{\mathbb Z}
\newcommand{\F}{\mathbb F}
\newcommand{\Q}{\mathbb Q}
%\newcommand{\C}{\mathbb C}
\DeclareMathOperator{\Tr}{Tr}

% operators that should be typeset in Roman font
\newcommand{\poly}{\mathrm{poly}}
\newcommand{\polylog}{\mathrm{polylog}}
\newcommand{\sgn}{\mathrm{sgn}}
\newcommand{\avg}{\mathop{\mathrm{avg}}}
\newcommand{\val}{{\mathrm{val}}}

% complexity classes
\renewcommand{\P}{\mathrm{P}}
\newcommand{\NP}{\mathrm{NP}}
\newcommand{\BPP}{\mathrm{BPP}}
\newcommand{\DTIME}{\mathrm{DTIME}}
\newcommand{\ZPTIME}{\mathrm{ZPTIME}}
\newcommand{\BPTIME}{\mathrm{BPTIME}}
\newcommand{\NTIME}{\mathrm{NTIME}}

% values associated to optimization algorithm instances
\newcommand{\Opt}{{\mathsf{Opt}}}
\newcommand{\Alg}{{\mathsf{Alg}}}
\newcommand{\Lp}{{\mathsf{Lp}}}
\newcommand{\Sdp}{{\mathsf{Sdp}}}
\newcommand{\Exp}{{\mathsf{Exp}}}

% if you think the sum and product signs are too big in your math mode; x convention
% as in the probability operators
\newcommand{\littlesum}{{\textstyle \sum}}
\newcommand{\littlesumx}{\mathop{{\textstyle \sum}}}
\newcommand{\littleprod}{{\textstyle \prod}}
\newcommand{\littleprodx}{\mathop{{\textstyle \prod}}}

% horizontal line across the page
\newcommand{\horz}{
\vspace{-.4in}
\begin{center}
\begin{tabular}{p{\textwidth}}\\
\hline
\end{tabular}
\end{center}
}

% calligraphic letters
\newcommand{\calA}{{\cal A}}
\newcommand{\calB}{{\cal B}}
\newcommand{\calC}{{\cal C}}
\newcommand{\calD}{{\cal D}}
\newcommand{\calE}{{\cal E}}
\newcommand{\indep}{\rotatebox[origin=c]{90}{$\models$}}
\newcommand{\calF}{{\cal F}}
\newcommand{\calG}{{\cal G}}
\newcommand{\calH}{{\cal H}}
\newcommand{\calI}{{\cal I}}
\newcommand{\calJ}{{\cal J}}
\newcommand{\calK}{{\cal K}}
\newcommand{\calL}{{\cal L}}
\newcommand{\calM}{{\cal M}}
\newcommand{\calN}{{\cal N}}
\newcommand{\calO}{{\cal O}}
\newcommand{\calP}{{\cal P}}
\newcommand{\calQ}{{\cal Q}}
\newcommand{\calR}{{\cal R}}
\newcommand{\calS}{{\cal S}}
\newcommand{\calT}{{\cal T}}
\newcommand{\calU}{{\cal U}}
\newcommand{\calV}{{\cal V}}
\newcommand{\calW}{{\cal W}}
\newcommand{\calX}{{\cal X}}
\newcommand{\calY}{{\cal Y}}
\newcommand{\calZ}{{\cal Z}}
\newcommand{\com}{^\complement}
\newcommand{\inv}{^{-1}}
\newcommand{\tp}{^\top}
% bold letters (useful for random variables)
\renewcommand{\a}{{\boldsymbol a}}
\renewcommand{\b}{{\boldsymbol b}}
\renewcommand{\c}{{\boldsymbol c}}
\renewcommand{\d}{{\boldsymbol d}}
\newcommand{\e}{{\boldsymbol e}}
\newcommand{\f}{{\boldsymbol f}}
\newcommand{\g}{{\boldsymbol g}}
\newcommand{\h}{{\boldsymbol h}}
\renewcommand{\i}{{\boldsymbol i}}
\renewcommand{\j}{{\boldsymbol j}}
\renewcommand{\k}{{\boldsymbol k}}
\newcommand{\m}{{\boldsymbol m}}
\newcommand{\n}{{\boldsymbol n}}
\renewcommand{\o}{{\boldsymbol o}}
\newcommand{\p}{{\boldsymbol p}}
\newcommand{\q}{{\boldsymbol q}}
\renewcommand{\r}{{\boldsymbol r}}
\newcommand{\s}{{\boldsymbol s}}
\renewcommand{\t}{{\boldsymbol t}}
\renewcommand{\u}{{\boldsymbol u}}
\renewcommand{\v}{{\boldsymbol v}}
\newcommand{\w}{{\boldsymbol w}}
\newcommand{\x}{{\boldsymbol x}}
\newcommand{\y}{{\boldsymbol y}}
\newcommand{\z}{{\boldsymbol z}}
\newcommand{\A}{{\boldsymbol A}}
\newcommand{\B}{{\boldsymbol B}}
\newcommand{\D}{{\boldsymbol D}}
%\newcommand{\G}{{\boldsymbol G}}
\renewcommand{\H}{{\boldsymbol H}}
\newcommand{\I}{{\boldsymbol I}}
\newcommand{\J}{{\boldsymbol J}}
\newcommand{\K}{{\boldsymbol K}}
\renewcommand{\L}{{\boldsymbol L}}
\newcommand{\M}{{\boldsymbol M}}
\renewcommand{\O}{{\boldsymbol O}}
\renewcommand{\S}{{\boldsymbol S}}
\newcommand{\T}{{\boldsymbol T}}
%\newcommand{\U}{{\boldsymbol U}}
\newcommand{\V}{{\boldsymbol V}}
\newcommand{\W}{{\boldsymbol W}}
\newcommand{\X}{{\boldsymbol X}}
\newcommand{\Y}{{\boldsymbol Y}}
\newcommand{\bra}[1]{\left(#1\right)}
\newcommand{\abs}[1]{\left|#1\right|}
\newcommand{\innprod}[1]{\left\langle#1\right\rangle}
\newcommand{\norm}[1]{\left\|#1\right\|}
\newcommand{\wh}[1]{\widehat{#1}}
\newcommand{\wt}[1]{\widetilde{#1}}
\def\beq{\begin{equation}}
\def\eeq{\end{equation}}
\def\bal{\begin{aligned}}
\def\eal{\end{aligned}}
\def\beqal{\begin{equation}\begin{aligned}}
\def\eeqal{\end{aligned}\end{equation}}


% useful for Fourier analysis
\newcommand{\bits}{\{-1,1\}}
\newcommand{\bitsn}{\{-1,1\}^n}
\newcommand{\bn}{\bitsn}
\newcommand{\isafunc}{{: \bitsn \rightarrow \bits}}
\newcommand{\fisafunc}{{f : \bitsn \rightarrow \bits}}
\newcommand{\ndone}{\frac{1}{n}}

\makeatletter
\newcommand{\distas}[1]{\mathbin{\overset{#1}{\kern\z@\sim}}}%
\newsavebox{\mybox}\newsavebox{\mysim}
\newcommand{\distras}[1]{%
  \savebox{\mybox}{\hbox{\kern3pt$\scriptstyle#1$\kern3pt}}%
  \savebox{\mysim}{\hbox{$\sim$}}%
  \mathbin{\overset{#1}{\kern\z@\resizebox{\wd\mybox}{\ht\mysim}{$\sim$}}}%
}
\makeatother
\newcommand{\iid}{\distras{\text{i.i.d.}}}
\makeatletter
\newcommand*{\rom}[1]{\expandafter\@slowromancap\romannumeral #1@}
\makeatother

% if you want
\newcommand{\half}{{\textstyle \frac12}}

\newcommand{\myfig}[4]{\begin{figure}[h] \begin{center} \includegraphics[width=#1\textwidth]{#2} \caption{#3} \label{#4} \end{center} \end{figure}} 
\newcommand{\chara}{\mathbb{1}}

\Scribe{Xinze Li}
\Lecturer{Chao Gao}
\LectureNumber{09}
\LectureDate{Feb 05, 2020}
\LectureTitle{Precision Matrix Inference}

\lstset{style=mystyle}



\begin{document}
	\MakeScribeTop

%#############################################################
%#############################################################
%#############################################################
%#############################################################
Suppose that $X\sim \calN\bra{0,\Omega^{-1}}$ and that $\Omega = \Sigma^{-1}$ is the precision matrix. And thus
\beq
X=
\begin{pmatrix}
x_1\\
x_2
\end{pmatrix}
\sim \calN\bra{0, \begin{pmatrix}
\Omega_{11}&\Omega_{12}\\
\Omega_{21}&\Omega_{22}\\
\end{pmatrix}^{-1}}
\eeq
And so we have the \href{https://stats.stackexchange.com/questions/30588/deriving-the-conditional-distributions-of-a-multivariate-normal-distribution}{conditional probability distribution} as follows
\beq
x_1|x_2\sim \calN\bra{-\Omega_{11}^{-1}\Omega_{12}x_2, \Omega_{11}^{-1}}
\eeq
Suppose that $A={1,2}$, and that $A^\complement=\left\{ 3,\cdots, p\right\}$. So
\beq
X=\begin{pmatrix}
x_1\\
\vrule\\
x_p
\end{pmatrix}
=\begin{pmatrix}
X_A\\
X_{A^\complement}
\end{pmatrix}
\eeq
Thus
\beq\label{eq:defxa}
	X_A|X_{A\com}\sim \calN\bra{-\Omega_{AA}^{-1}\Omega_{AA\com}X_{A\com}, \Omega_{AA}^{-1}}
\eeq
Let
\beq\label{eq:defb}
B^\top = -\Omega_{AA}\inv \Omega_{AA\com},\quad B\in\R^{(p-2)\times 2}
\eeq
Now
\beq
\Omega_{AA}=
\begin{pmatrix}
\Omega_{11}&\Omega_{12}\\
\Omega_{21}&\Omega_{22}
\end{pmatrix}
\eeq
We suppose that $X_1, \cdots, X_n \iid \calN(0, \Omega\inv)$ and that
\beq
\max_j \sum_k \chara\bra{\Omega_{jk}\ne 0}\leq s
\eeq
Also suppose
\beq
M\inv\leq \lam_{\min} (\Omega)\leq \lam_{\max} (\Omega)\leq M
\eeq
and that $\wh{B}$ is the  lasso estimate
\beq
\wh{B}=\arg\min_{B\in\R^{(p-2)\times 2}}\sum_i\norm{X_{iA}-B\tp X_{iA\com}}^2+\lam\norm{B}_1
\eeq
By HW6 P2, we know that w.h.p. the following holds
\begin{equation}\bal\label{eq:lassoprop}
\frac{1}{n} \sum_{i=1}^{n}\left\|\left(\widehat{B}-B\right)^{T} X_{i A^{\complement}}\right\|^{2} &\lesssim \frac{s \log p}{n}\\
\left\|\widehat{B}-B\right\|_{1} &\lesssim s \sqrt{\frac{\log p}{n}}
\eal\end{equation}
$B$ is defined in ~\ref{eq:defb}. We now define the estimator
\beq\label{eq:OAA}
\wh{\Omega}_{AA}\inv = \ndone \sum_i \bra{X_{iA}-\wh{B}\tp X_{iA\com}}\cdot\bra{X_{iA}-\wh{B}\tp X_{iA\com}}\tp
\eeq
Now note that from~\ref{eq:defxa}, we could write
\beq\label{eq:xia}
X_{iA}=B\tp X_{iA\com}+\Omega_{AA}^{-\frac{1}{2}}w_i, \quad w_i\sim \calN(0, I_2)
\eeq
Combining \cref{eq:OAA,eq:xia}, we have
\beqal
\wh{\Omega}_{AA}\inv& = \ndone \sum_i \bra{B\tp X_{iA\com}+\Omega_{AA}^{-\frac{1}{2}}w_i-\wh{B}\tp X_{iA\com}}\cdot\bra{B\tp X_{iA\com}+\Omega_{AA}^{-\frac{1}{2}}w_i-\wh{B}\tp X_{iA\com}}\tp\\
&=\underbrace{\ndone\sum_i \Omega_{AA}^{-\frac{1}{2}}w_iw_i\tp\Omega_{AA}^{-\frac{1}{2}}}_{(\text{\rom{1}})}+ \underbrace{\ndone\sum_i \bra{\bra{B-\wh{B}}\tp X_{iA\com}}\cdot \bra{\bra{B-\wh{B}}\tp X_{iA\com}}\tp}_{(\text{\rom{2}})}\\
&+\underbrace{\ndone\sum_i \bra{\Omega_{AA}^{-\frac{1}{2}}w_i}\cdot X_{iA\com}\tp \bra{B-\wh{B}} }_{(\text{\rom{3}})}+\underbrace{\ndone\sum_i \bra{B-\wh{B}}\tp X_{iA\com} \cdot  \bra{\Omega_{AA}^{-\frac{1}{2}}w_i}\tp }_{(\text{\rom{4}})}
\eeqal
Note that
\beq
(\text{\rom{1}})-\Omega_{AA}^{-1}=\ndone\Omega_{AA}^{-\frac{1}{2}}\bra{\sum_i w_iw_i\tp-I_2}\Omega_{AA}^{-\frac{1}{2}}
\eeq
which is asymptotically normal.
For $(\text{\rom{2}})$, from ~\ref{eq:lassoprop}, we have
\beq
\norm{(\text{\rom{2}})}_{op}\leq \frac{1}{n} \sum_{i=1}^{n}\left\|\left(\widehat{B}-B\right)^{T} X_{i A^{\complement}}\right\|^{2} \lesssim \frac{s \log p}{n}\\
\eeq
The magnitude of $(\text{\rom{3}})$ and $(\text{\rom{4}})$ is the same, so we simply analyze $\norm{(\text{\rom{3}})}_F$
\beqal
\norm{(\text{\rom{3}})}_F&= \norm{\bra{B-\wh{B}}^\top\cdot \bra{\ndone\sum_i X_{iA\com} \bra{\Omega_{AA}^{-\frac{1}{2}}w_i}\tp }}_F\\
&\leq \norm{B-\wh{B}}_1\cdot \norm{\ndone\sum_i X_{iA\com} \bra{\Omega_{AA}^{-\frac{1}{2}}w_i}\tp}_\infty\\
&\lesssim s\sqrt{\frac{\log p}{n}}\cdot \max_{j\in [2],k\in [p-2]}\ndone\sum_i \bra{\Omega_{AA}^{-\frac{1}{2}}w_i}_j\cdot \bra{X_{iA\com}}_k\\
&\lesssim s\sqrt{\frac{\log p}{n}}\cdot \sqrt{\frac{\log p}{n}}=s\frac{\log p}{n}
\eeqal
Thus
\beq
\sqrt{n} \bra{\wh{\Omega}_{AA}\inv -\Omega_{AA}\inv} = \frac{1}{\sqrt{n}}\Omega_{AA}^{-\frac{1}{2}}\bra{\sum_i w_iw_i\tp-I_2}\Omega_{AA}^{-\frac{1}{2}}+O_p\bra{\frac{s\log p}{\sqrt{n}}}
\eeq
We conclude with the following theorem (See HW6 2(c) for details)
\begin{theorem}[\cite{Ren_2015}]
If $\frac{s\log p}{\sqrt{n}}\rightarrow 0$, then
\beq
\sqrt{n}\bra{\wh{\Omega}_{12}-\Omega_{12}}\leadsto \calN\bra{0, \Omega_{11}\Omega_{22}+\Omega_{12}^2}
\eeq
\end{theorem}
From \cite{Ren_2015}, define parameter space
\begin{equation}
\mathcal{G}_{0}\left(M, k_{n, p}\right)=\left\{\begin{aligned} \Omega=\left(\omega_{i j}\right)_{1 \leq i, j \leq p}: \max _{1 \leq j \leq p} \sum_{i=1}^{p} 1\left\{\omega_{i j} \neq 0\right\} \leq k_{n, p} \\ \text { and } 1 / M \leq \lambda_{\min }(\Omega) \leq \lambda_{\max }(\Omega) \leq M \end{aligned}\right\}
\end{equation}
 we also know the minimax rate
 \begin{equation}
 \inf _{\hat{\omega}_{i j}} \sup _{\mathcal{G}_{0}\left(M, k_{n, p}\right)} \mathbb{E}\left|\hat{\omega}_{i j}-\omega_{i j}\right| \asymp \max \left\{n^{-1} k_{n, p} \log p, n^{-1 / 2}\right\}
 \end{equation}
Thus we could observe that ther condition $\frac{s\log p}{\sqrt{n}}$ is necessary.





\bibliographystyle{apalike}
\bibliography{PrecisionInference.bib}	
%%%%%%%%%%% end of doc
\end{document}
