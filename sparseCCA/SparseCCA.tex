\documentclass[12pt]{article}
\usepackage[english]{babel}
\usepackage[utf8x]{inputenc}
\usepackage[T1]{fontenc}
\usepackage{scribe}
\usepackage{listings}
\usepackage{algorithm}
\usepackage{mathtools}
\usepackage{algorithmicx}
\usepackage{algpseudocode}
\usepackage{fancyhdr}
\usepackage{extramarks}
\usepackage{hyperref}
\usepackage{cleveref}
%\usepackage[style=authoryear]{biblatex}
\renewcommand{\Pr}{{\bf Pr}}
\newcommand{\Prx}{\mathop{\bf Pr\/}}
\newcommand{\E}{{\bf E}}
\newcommand{\Ex}{\mathop{\bf E\/}}
\newcommand{\Var}{{\bf Var}}
\newcommand{\Varx}{\mathop{\bf Var\/}}
\newcommand{\Cov}{{\bf Cov}}
\newcommand{\Cor}{{\bf Corr}}
\newcommand{\Covx}{\mathop{\bf Cov\/}}
\newcommand{\rank}{\text{rank}}

% shortcuts for symbol names that are too long to type
\newcommand{\eps}{\epsilon}
\newcommand{\lam}{\lambda}
\renewcommand{\l}{\ell}
\newcommand{\la}{\langle}
\newcommand{\ra}{\rangle}
%\newcommand{\wh}{\widehat}

% "blackboard-fonted" letters for the reals, naturals etc.
\newcommand{\R}{\mathbb R}
\newcommand{\N}{\mathbb N}
\newcommand{\Z}{\mathbb Z}
\newcommand{\F}{\mathbb F}
\newcommand{\Q}{\mathbb Q}
%\newcommand{\C}{\mathbb C}
\DeclareMathOperator{\Tr}{Tr}

% operators that should be typeset in Roman font
\newcommand{\poly}{\mathrm{poly}}
\newcommand{\polylog}{\mathrm{polylog}}
\newcommand{\sgn}{\mathrm{sgn}}
\newcommand{\avg}{\mathop{\mathrm{avg}}}
\newcommand{\val}{{\mathrm{val}}}

% complexity classes
\renewcommand{\P}{\mathrm{P}}
\newcommand{\NP}{\mathrm{NP}}
\newcommand{\BPP}{\mathrm{BPP}}
\newcommand{\DTIME}{\mathrm{DTIME}}
\newcommand{\ZPTIME}{\mathrm{ZPTIME}}
\newcommand{\BPTIME}{\mathrm{BPTIME}}
\newcommand{\NTIME}{\mathrm{NTIME}}

% values associated to optimization algorithm instances
\newcommand{\Opt}{{\mathsf{Opt}}}
\newcommand{\Alg}{{\mathsf{Alg}}}
\newcommand{\Lp}{{\mathsf{Lp}}}
\newcommand{\Sdp}{{\mathsf{Sdp}}}
\newcommand{\Exp}{{\mathsf{Exp}}}

% if you think the sum and product signs are too big in your math mode; x convention
% as in the probability operators
\newcommand{\littlesum}{{\textstyle \sum}}
\newcommand{\littlesumx}{\mathop{{\textstyle \sum}}}
\newcommand{\littleprod}{{\textstyle \prod}}
\newcommand{\littleprodx}{\mathop{{\textstyle \prod}}}

% horizontal line across the page
\newcommand{\horz}{
\vspace{-.4in}
\begin{center}
\begin{tabular}{p{\textwidth}}\\
\hline
\end{tabular}
\end{center}
}

% calligraphic letters
\newcommand{\calA}{{\cal A}}
\newcommand{\calB}{{\cal B}}
\newcommand{\calC}{{\cal C}}
\newcommand{\calD}{{\cal D}}
\newcommand{\calE}{{\cal E}}
\newcommand{\indep}{\rotatebox[origin=c]{90}{$\models$}}
\newcommand{\calF}{{\cal F}}
\newcommand{\calG}{{\cal G}}
\newcommand{\calH}{{\cal H}}
\newcommand{\calI}{{\cal I}}
\newcommand{\calJ}{{\cal J}}
\newcommand{\calK}{{\cal K}}
\newcommand{\calL}{{\cal L}}
\newcommand{\calM}{{\cal M}}
\newcommand{\calN}{{\cal N}}
\newcommand{\calO}{{\cal O}}
\newcommand{\calP}{{\cal P}}
\newcommand{\calQ}{{\cal Q}}
\newcommand{\calR}{{\cal R}}
\newcommand{\calS}{{\cal S}}
\newcommand{\calT}{{\cal T}}
\newcommand{\calU}{{\cal U}}
\newcommand{\calV}{{\cal V}}
\newcommand{\calW}{{\cal W}}
\newcommand{\calX}{{\cal X}}
\newcommand{\calY}{{\cal Y}}
\newcommand{\calZ}{{\cal Z}}
\newcommand{\com}{^\complement}
\newcommand{\inv}{^{-1}}
\newcommand{\tp}{^\top}
% bold letters (useful for random variables)
\renewcommand{\a}{{\boldsymbol a}}
\renewcommand{\b}{{\boldsymbol b}}
\renewcommand{\c}{{\boldsymbol c}}
\renewcommand{\d}{{\boldsymbol d}}
\newcommand{\e}{{\boldsymbol e}}
\newcommand{\f}{{\boldsymbol f}}
\newcommand{\g}{{\boldsymbol g}}
\newcommand{\h}{{\boldsymbol h}}
\renewcommand{\i}{{\boldsymbol i}}
\renewcommand{\j}{{\boldsymbol j}}
\renewcommand{\k}{{\boldsymbol k}}
\newcommand{\m}{{\boldsymbol m}}
\newcommand{\n}{{\boldsymbol n}}
\renewcommand{\o}{{\boldsymbol o}}
\newcommand{\p}{{\boldsymbol p}}
\newcommand{\q}{{\boldsymbol q}}
\renewcommand{\r}{{\boldsymbol r}}
\newcommand{\s}{{\boldsymbol s}}
\renewcommand{\t}{{\boldsymbol t}}
\renewcommand{\u}{{\boldsymbol u}}
\renewcommand{\v}{{\boldsymbol v}}
\newcommand{\w}{{\boldsymbol w}}
\newcommand{\x}{{\boldsymbol x}}
\newcommand{\y}{{\boldsymbol y}}
\newcommand{\z}{{\boldsymbol z}}
\newcommand{\A}{{\boldsymbol A}}
\newcommand{\B}{{\boldsymbol B}}
\newcommand{\D}{{\boldsymbol D}}
%\newcommand{\G}{{\boldsymbol G}}
\renewcommand{\H}{{\boldsymbol H}}
\newcommand{\I}{{\boldsymbol I}}
\newcommand{\J}{{\boldsymbol J}}
\newcommand{\K}{{\boldsymbol K}}
\renewcommand{\L}{{\boldsymbol L}}
\newcommand{\M}{{\boldsymbol M}}
\renewcommand{\O}{{\boldsymbol O}}
\renewcommand{\S}{{\boldsymbol S}}
\newcommand{\T}{{\boldsymbol T}}
%\newcommand{\U}{{\boldsymbol U}}
\newcommand{\V}{{\boldsymbol V}}
\newcommand{\W}{{\boldsymbol W}}
\newcommand{\X}{{\boldsymbol X}}
\newcommand{\Y}{{\boldsymbol Y}}
\newcommand{\bra}[1]{\left(#1\right)}
\newcommand{\bgbra}[1]{\left\{#1\right\}}
\newcommand{\abs}[1]{\left|#1\right|}
\newcommand{\innprod}[1]{\left\langle#1\right\rangle}
\newcommand{\norm}[1]{\left\|#1\right\|}
\newcommand{\wh}[1]{\widehat{#1}}
\newcommand{\wt}[1]{\widetilde{#1}}
\newcommand{\srt}{^{\frac{1}{2}}}
\newcommand{\idx}[1]{^{\bra{#1}}}
\def\beq{\begin{equation}}
\def\eeq{\end{equation}}
\def\bal{\begin{aligned}}
\def\eal{\end{aligned}}
\def\beqal{\begin{equation}\begin{aligned}}
\def\eeqal{\end{aligned}\end{equation}}
\newcommand{\mtline}[1]{\beq
\left\{
\bal
#1
\eal
\right.
\eeq}


% useful for Fourier analysis
\newcommand{\bits}{\{-1,1\}}
\newcommand{\bitsn}{\{-1,1\}^n}
\newcommand{\bn}{\bitsn}
\newcommand{\isafunc}{{: \bitsn \rightarrow \bits}}
\newcommand{\fisafunc}{{f : \bitsn \rightarrow \bits}}
\newcommand{\ndone}{\frac{1}{n}}
\newcommand{\supp}{\text{supp}}
\newcommand*\Let[2]{\State #1 $\gets$ #2}


\makeatletter
\newcommand{\distas}[1]{\mathbin{\overset{#1}{\kern\z@\sim}}}%
\newsavebox{\mybox}\newsavebox{\mysim}
\newcommand{\distras}[1]{%
  \savebox{\mybox}{\hbox{\kern3pt$\scriptstyle#1$\kern3pt}}%
  \savebox{\mysim}{\hbox{$\sim$}}%
  \mathbin{\overset{#1}{\kern\z@\resizebox{\wd\mybox}{\ht\mysim}{$\sim$}}}%
}
\makeatother
\newcommand{\iid}{\distras{\text{i.i.d.}}}
\makeatletter
\newcommand*{\rom}[1]{\expandafter\@slowromancap\romannumeral #1@}
\makeatother

% if you want
\newcommand{\half}{{\textstyle \frac12}}

\newcommand{\myfig}[4]{\begin{figure}[h] \begin{center} \includegraphics[width=#1\textwidth]{#2} \caption{#3} \label{#4} \end{center} \end{figure}} 
\newcommand{\chara}{\mathbb{1}}

\Scribe{Xinze Li}
\Lecturer{Chao Gao}
\LectureNumber{12}
\LectureDate{Feb 17, 2020}
\LectureTitle{Sparse CCA}

\lstset{style=mystyle}



\begin{document}
	\MakeScribeTop
\section{Intro to Sparse CCA}
Let's first describe the problem setting. Let $X\in \R^p$, and $Y\in \R^q$. Our goal is to find $\theta$ and $\eta$ such that $\Cor\bra{\theta^\top X, \eta^\top Y}$ is the largest. Suppose
\beq
\Cov\begin{pmatrix}
X\\Y
\end{pmatrix}=
\begin{pmatrix}
\Sigma_{x}&\Sigma_{xy}\\
\Sigma_{yx}&\Sigma_{y}\\
\end{pmatrix}
\eeq
Then we have
\beq
\Cor\bra{\theta^\top X, \eta^\top Y}=\frac{\theta^\top \Sigma_{xy}\eta}{\sqrt{\theta^\top \Sigma_{x}\theta}\sqrt{\eta^\top \Sigma_{y}\eta}}
\eeq
We could formulate the problem as
\beqal
\max\quad& \theta^\top \Sigma_{xy}\eta\\
s.t.\quad& \theta^\top \Sigma_{x}\theta=\eta^\top \Sigma_{y}\eta=1
\eeqal
Suppose that
\beq\label{eq:transform}
\left\{
\bal
a&=\Sigma_{xx}^{\frac{1}{2}}\theta\\
b&=\Sigma_{yy}^{\frac{1}{2}}\eta\\
\eal
\right.
\eeq
So now the problem is
\beqal
\max\quad& a^\top\Sigma_{x}^{-\frac{1}{2}} \Sigma_{xy}\Sigma_{y}^{-\frac{1}{2}}b\\
s.t.\quad& \norm{a}=\norm{b}=1
\eeqal
If we do SVD on $\Sigma_{x}^{-\frac{1}{2}} \Sigma_{xy}\Sigma_{y}^{-\frac{1}{2}}$, i.e.
\beq
\Sigma_{xx}^{-\frac{1}{2}} \Sigma_{xy}\Sigma_{yy}^{-\frac{1}{2}}=\sum_j \lam_ja_jb_j^\top
\eeq
Then the solution of $a$ and $b$ is $(a_1,b_1)$. And the solution of $\theta$ and $\eta$ is $(\Sigma_{x}^{-\frac{1}{2}}a_1,\Sigma_{y}^{-\frac{1}{2}}b_1)$. And so by~\ref{eq:transform} we have
\beq
\Sigma_{xy}=\Sigma_{x}\bra{\sum_j \lam_j\theta_j\eta_j^\top}\Sigma_{y}
\eeq
Note that 
\beq
\theta_j^\top\Sigma_{x}\theta_k=\eta_j^\top\Sigma_{y}\eta_k=\delta_j^k
\eeq
\section{Problem Formulation}
For sparse CCA, we only consider rank 1 model, i.e. $\text{rank}(\Sigma_{xy})=1$ and
\beq\label{eq:sigmaxy}
\Sigma_{xy}=\Sigma_{x}\bra{ \lam\theta\eta^\top}\Sigma_{y}
\eeq
Also suppose that
\beq
\theta^\top \Sigma_{x}\theta=\eta^\top \Sigma_{y}\eta=1
\eeq
and that $\theta\in\Theta(p,s_1)$ and $\eta\in\Theta(p,s_2)$. We further assume that
\beqal
M\inv \leq \lam_{\min}(\Sigma_{x})\leq \lam_{\max}\bra{\Sigma_{x}}\leq M\\
M\inv \leq \lam_{\min}(\Sigma_{y})\leq \lam_{\max}\bra{\Sigma_{y}}\leq M
\eeqal
Now suppose that 
\beq
\begin{pmatrix}
X_1\\Y_1
\end{pmatrix}
,\cdots,\begin{pmatrix}
X_n\\Y_n
\end{pmatrix}
\iid \calN\bra{\begin{pmatrix}
0\\0
\end{pmatrix}
,\begin{pmatrix}
\Sigma_{x}&\Sigma_{xy}\\
\Sigma_{yx}&\Sigma_{y}
\end{pmatrix}}
\eeq
And a natural estimator is the following
\beqal
\max_{\theta\in \R^p, \eta\in \R^q} \quad &\theta\tp \wh{\Sigma}_{xy}\eta\\
s.t.\quad &\theta\tp \wh{\Sigma}_x\theta =1,\quad \eta\tp \wh{\Sigma}_{y}\eta =1\\
&\norm{\theta}_0=s_1,\quad \norm{\eta}_0\leq s_2
\eeqal
We know that this estimator achieves the minimax rate, but we will not analyze the property. Instead, let's consider the convex relaxation as sparse PCA.
\beq
\theta\tp \wh{\Sigma}_{xy}\eta = \innprod{\wh{\Sigma}_{xy}, A},\quad A = \theta\eta\tp
\eeq
Note that
\beq
\wh{\Sigma}_x^{\frac{1}{2}}A\wh{\Sigma}_y^{\frac{1}{2}}=\bra{\wh{\Sigma}_x\srt \theta}\cdot \bra{\wh{\Sigma}_y\srt \eta}\tp
\eeq
And so we could construct the following set
\beq
\calA = \bgbra{ab\tp:a\in\R^p,b\in\R^q,\norm{a}=\norm{b}=1}
\eeq
and obviously
\beq
\wh{\Sigma}_x^{\frac{1}{2}}A\wh{\Sigma}_y^{\frac{1}{2}}\in \calA
\eeq
Now we state the following lemma
\begin{lemma}
\beq \text{conv}(\calA)=\calB=\bgbra{B\in\R^{p\times q}:\norm{B}_N\leq 1}\eeq 
\end{lemma}
\begin{proof}
First, obviously, $\calA\subseteq\calB$. Now we need to prove that every element in $\calB$, it could be represented as convex combination of elements in $\calA$. Note that if $B=\sum\lam_ju_jv_j\tp$, and $\norm{B}_N\leq 1$, then $\sum_j\lam_j\leq 1$ and $\lam_j\geq 0$. Thus,
\beq
B=\sum_j \lam_ju_jv_j\tp+\bra{1-\sum_j \lam_j}\cdot 0
\eeq
Since $0\in\text{conv}(\calA)\subseteq \calB$, \footnote{Because if $ab\tp\in\calA$, then $-ab\tp\in\calA$, and $0=\frac{1}{2}ab\tp +\frac{1}{2}\bra{-ab\tp}$} the proof is completed.
\end{proof}
So now we could introduce the following convex program as in~\cite{gao2014sparse}
\beqal\label{eq:cvxrelax}
\max_{A\in\R^{p\times q}}\quad &\innprod{\wh{\Sigma}_{xy}, A}-\rho \norm{A}_1\\
s.t.\quad &\norm{\wh{\Sigma}_x\srt A\wh{\Sigma}_y\srt}_N\leq 1
\eeqal
\section{Analysis}
Now note that we could not use~\ref{eq:sigmaxy} directly to basic inequality directly, because it may not be feasible to the convex constraint. Thus we now introduce
\beq
\wt{\theta}=\frac{\theta}{\sqrt{\theta\tp\wh{\Sigma}_x\theta}},\quad \wt{\eta}=\frac{\eta}{\sqrt{\eta\tp\wh{\Sigma}_y\eta}}
\eeq
and we have
\beq
\norm{\wh{\Sigma}_x\srt \wt{\theta}}=1,\quad \norm{\wh{\Sigma}_y\srt \wt{\eta}}=1
\eeq
Now we define the surrogate of the truth $\wt{\Sigma}$ as
\beq\label{eq:Atilde}
\wt{A}=\wt{\theta}\wt{\eta}\tp
\eeq
which satisfies the constraint
\beq
\norm{\wh{\Sigma}_x\srt \wt{A}\wh{\Sigma}_y\srt}_N\leq 1
\eeq
So suppose $\wh{A}$ is the solution to the convex relaxation problem in~\ref{eq:cvxrelax},
 now we could apply the basic inequality
\beq\label{eq:basic}
\innprod{\wh{\Sigma}_{xy}, \wh{A}}-\rho \norm{\wh{A}}_1\geq \innprod{\wh{\Sigma}_{xy}, \wt{A}}-\rho\norm{\wt{A}}_1
\eeq
Naturally, we also introduce
\beq
\wt{\Sigma}_{xy}=\wh{\Sigma}_x(\lam\theta \eta\tp)\wh{\Sigma}_y
\eeq
Rearranging the order of~\ref{eq:basic}, we have
\beq
\innprod{\wt{\Sigma}_{xy}, \wt{A}-\wh{A}}\leq \rho \bra{\norm{\wt{A}}_1-\norm{\wh{A}}_1}+\innprod{\wt{\Sigma}_{xy}-\wh{\Sigma}_{xy}, \wt{A}-\wh{A}}
\eeq
Using the following notation
\beqal
&\Delta = \wh{A}-\wt{A},\quad S_1=\text{supp}(\theta),\quad S_2=\supp(\eta)\\
&\norm{\Delta_{S_1S_2}}=\sum_{(j,k)\in S_1\times S_2}\abs{\Delta_{jk}},\quad \norm{\Delta_{(S_1S_2)\com}}=\sum_{(j,k)\in (S_1\times S_2)\com}\abs{\Delta_{jk}}
\eeqal
and applying Holder's inequality gives
\beq
\innprod{\wt{\Sigma}_{xy}, \wt{A}-\wh{A}}\leq \rho \bra{\norm{\wt{A}}_1-\norm{\wh{A}}_1}+\norm{ \wt{\Sigma}_{xy}-\wh{\Sigma}_{xy} }_\infty\cdot \norm{ \wt{A}-\wh{A} }_1
\eeq
Now suppose $\norm{ \wt{\Sigma}_{xy}-\wh{\Sigma}_{xy} }_\infty\leq \frac{\rho}{2}$, then
\beqal\label{eq:anal1}
\innprod{\wt{\Sigma}_{xy}, \wt{A}-\wh{A}}&\leq \rho\bra{\norm{\wt{A}_{S_1S_2}}_1 -\norm{\wt{A}_{S_1S_2} +\Delta_{S_1S_2} }_1- \norm{\Delta_{(S_1S_2)\com}}_1}\\
&+\frac{\rho}{2}\norm{\Delta_{S_1S_2}}_1+\frac{\rho}{2}\norm{\Delta_{(S_1S_2)\com}}_1\\
&\leq \frac{3\rho}{2}\norm{\Delta_{S_1S_2}}-\frac{\rho}{2}\norm{\Delta_{(S_1S_2)\com}}_1
\eeqal
\subsection{Cone Condition}
Now we need to show that the LHS is nonnegative in order to obtain the cone condition. So we want to relate it to a loss function. First let's normalize $\wt{\Sigma}_{xy}$
\beq
\wt{\Sigma}_{xy}=\wh{\Sigma}_x(\lam\theta\eta\tp)\wh{\Sigma}_y=\wh{\Sigma}_x(\wt{\lam}\wt{\theta}\wt{\eta}\tp)\wh{\Sigma}_y
\eeq
where $\wt{\lam}$ is defined as
\beq
\wt{\lam}=\lam \sqrt{\theta\tp\wh{\Sigma}_x\theta}\sqrt{\eta\tp\wh{\Sigma}_y\eta}=(1+o_p(1))\lam
\eeq
Then we expand the innerproduct term $\innprod{\wt{\Sigma}_{xy}, \wt{A}-\wh{A}}$
\beqal
\innprod{\wt{\Sigma}_{xy}, \wt{A}-\wh{A}}&=\wt{\lam}\innprod{\wh{\Sigma}_x\wt{\theta}\wt{\eta}\tp \wh{\Sigma}_y,\wt{\theta}\wt{\eta}\tp-\wh{A}}\\
&=\wt{\lam}\bra{1-\innprod{ \wh{\Sigma}_x\srt\wt{\theta}\wt{\eta}\tp \wh{\Sigma}_y\srt,  \wh{\Sigma}_x\srt\wh{A}\wh{\Sigma}_y\srt}}
\eeqal
Now note that
\beqal
\norm{\wh{\Sigma}_x\srt\wh{A}\wh{\Sigma}_y\srt- \wh{\Sigma}_x\srt\wt{\theta}\wt{\eta}\tp \wh{\Sigma}_y\srt }_F^2 &= \norm{\wh{\Sigma}_x\srt\wh{A}\wh{\Sigma}_y\srt}_F^2+\norm{\wh{\Sigma}_x\srt\wt{\theta}\wt{\eta}\tp \wh{\Sigma}_y\srt}_F^2-2\innprod{ \wh{\Sigma}_x\srt\wt{\theta}\wt{\eta}\tp \wh{\Sigma}_y\srt,  \wh{\Sigma}_x\srt\wh{A}\wh{\Sigma}_y\srt}\\
&=1+\norm{\wh{\Sigma}_x\srt\wh{A}\wh{\Sigma}_y\srt}_F^2-2\innprod{ \wh{\Sigma}_x\srt\wt{\theta}\wt{\eta}\tp \wh{\Sigma}_y\srt,  \wh{\Sigma}_x\srt\wh{A}\wh{\Sigma}_y\srt}\\
&\leq 2-2\innprod{ \wh{\Sigma}_x\srt\wt{\theta}\wt{\eta}\tp \wh{\Sigma}_y\srt,  \wh{\Sigma}_x\srt\wh{A}\wh{\Sigma}_y\srt}\\
\eeqal
Thus, the LHS of~\ref{eq:anal1} has the lower bound
\beq
\innprod{\wt{\Sigma}_{xy}, \wt{A}-\wh{A}}\geq \frac{\wt{\lam}}{2}\norm{\wh{\Sigma}_x\srt(\wh{A}-\wt{A})\wh{\Sigma}_y\srt}_F^2\geq 0
\eeq
So we now conclude that if $\rho \geq 2\norm{ \wt{\Sigma}_{xy}-\wh{\Sigma}_{xy} }_\infty$, then
\mtline{\label{eq:conecondition}
\frac{\wt{\lam}}{2}\norm{\wh{\Sigma}_x\srt(\wh{A}-\wt{A})\wh{\Sigma}_y\srt}_F^2&\leq \frac{3\rho }{2}\norm{\Delta}_F\\
\norm{\Delta_{(S_1S_2)\com}}_1&\leq 3\norm{\Delta_{S_1S_2}}_1
}
Note that the first inequality is derived because
\beq
\norm{\Delta_{S_1S_2}}_1\leq \sqrt{s_1s_2}\norm{\Delta_{S_1S_2}}_F\leq\sqrt{s_1s_2}\norm{\Delta}_F
\eeq
\subsection{Peeling}
In this subsection, we will use the peeling technique as in Lasso to compare $\norm{\wh{\Sigma}_x\srt\Delta\wh{\Sigma}_y\srt}_F$ with $\norm{\Delta}_F$. We will use the following notation for restricted eigenvalue
\beq
\lam_{\min}^\Gamma(k)=\min_{\norm{u}=1,\norm{u}_0\leq k}u\tp \Gamma u,\quad 
\lam_{\max}^\Gamma(k)=\max_{\norm{u}=1,\norm{u}_0\leq k}u\tp \Gamma u
\eeq
We will peel $p\times q$ into $K+1$ part according to $\Delta$. The first part is just $S_1\times S_2$ that contains $s_1s_2$ elements. The second part is denoted as $J_1$ as
\beq
J_1=\bgbra{(j,k):\abs{\Delta_{jk}}\text{ are the largest }t\text{ among all entries in }(S_1\times S_2)\com}
\eeq
and the rest as follows. Note that $\abs{J_K}\leq t$, and $\abs{J_k}=t$, for all $1\leq k\leq K$. Also denote 
\beq
\wt{J}=(S_1\times S_2)\cup J_1
\eeq
Using triangle inequality, we have
\beqal\label{eq:anal2}
\norm{\wh{\Sigma}_x\srt\Delta\wh{\Sigma}_y\srt}_F&\geq \norm{\wh{\Sigma}_x\srt\Delta_{\wt{J}}\wh{\Sigma}_y\srt}_F-\sum_{k=2}^{K}\norm{\wh{\Sigma}_x\srt\Delta_{J_k}\wh{\Sigma}_y\srt}_F\\
&\geq \sqrt{\lam_{\min}^{\wh{\Sigma}_x}(s_1s_2+t)\lam_{\min}^{\wh{\Sigma}_y}(s_1s_2+t)}\norm{\Delta_{\wt{J}}}_F-\sqrt{\lam_{\max}^{\wh{\Sigma}_x}(t)\lam_{\max}^{\wh{\Sigma}_y}(t)}\sum_{k=2}^K\norm{\Delta_{J_k}}_F
\eeqal
Now let's estimate $\sum_{k=2}^K \norm{\Delta_{J_k}}_F$, which is essentially the proof in Lasso
\beqal
\sum_{k=2}^K \norm{\Delta_{J_k}}_F&\leq \sum_{k=2}^K \sqrt{t}\norm{\Delta_{J_k}}_\infty\leq \sum_{k=2}^K \sqrt{t}\frac{\norm{\Delta_{J_{k-1}}}_1}{t}\\
&\leq \frac{1}{\sqrt{t}}\sum_{k=1}^K\norm{\Delta_{J_k}}_1 = \frac{1}{\sqrt{t}}\norm{\Delta_{(S_1S_2)\com}}_1\\
&\leq \frac{3}{\sqrt{t}}\norm{\Delta_{S_1S_2}}_1\leq 3\sqrt{\frac{s_1s_2}{t}}\norm{\Delta_{\wt{J}}}_F
\eeqal
Combining with~\ref{eq:anal2}, we have
\beq
\norm{\wh{\Sigma}_x\srt\Delta\wh{\Sigma}_y\srt}_F\geq \norm{\Delta_{\wt{J}}}_F\cdot\bra{ \sqrt{\lam_{\min}^{\wh{\Sigma}_x}(s_1s_2+t)\lam_{\min}^{\wh{\Sigma}_y}(s_1s_2+t)}-3\sqrt{\frac{s_1s_2}{t}} \sqrt{\lam_{\max}^{\wh{\Sigma}_x}(t)\lam_{\max}^{\wh{\Sigma}_y}(t)}} 
\eeq
Note that because
\beq
\norm{\Delta_{\wt{J}\com}}_F\leq \sum_{k=2}^K\norm{\Delta_{J_k}}_F\leq 3\sqrt{\frac{s_1s_2}{t}}\norm{\Delta_{\wt{J}}}_F
\eeq
we have
\beq
\norm{\Delta}_F\leq \norm{\Delta_{\wt{J}}}_F+\norm{\Delta_{\wt{J}\com}}_F\leq \bra{1+3\sqrt{\frac{s_1s_2}{t}}}\norm{\Delta_{\wt{J}}}_F
\eeq
Thus
\beq
\norm{\wh{\Sigma}_x\srt\Delta\wh{\Sigma}_y\srt}_F\geq \norm{\Delta}_F\cdot\bra{\frac{ \sqrt{\lam_{\min}^{\wh{\Sigma}_x}(s_1s_2+t)\lam_{\min}^{\wh{\Sigma}_y}(s_1s_2+t)}-3\sqrt{\frac{s_1s_2}{t}} \sqrt{\lam_{\max}^{\wh{\Sigma}_x}(t)\lam_{\max}^{\wh{\Sigma}_y}(t)}}{1+3\sqrt{\frac{s_1s_2}{t}}} } 
\eeq
So if we choose $t=Cs_1s_2$, then we could show that w.h.p.
\beq
\norm{\wh{\Sigma}_x\srt\Delta\wh{\Sigma}_y\srt}_F\gtrsim \norm{\Delta}_F
\eeq
as long as $\frac{s_1s_2\bra{\log p+\log q}}{n}$ is small. Combining with~\ref{eq:conecondition}, we now have
\beq
\norm{\Delta}_F^2\lesssim \frac{s_1s_2\rho^2}{\wt{\lam}^2}\lesssim \frac{s_1s_2}{\lam^2}\rho^2
\eeq
The rest is to choose $\rho$ such that
\beq
\rho \geq 2\norm{\wh{\Sigma}_{xy}-\wt{\Sigma}_{xy}}_\infty
\eeq
So now let's analyze $\norm{\wh{\Sigma}_{xy}-\wt{\Sigma}_{xy}}_\infty$
\beqal
\norm{\wh{\Sigma}_{xy}-\wt{\Sigma}_{xy}}_\infty&=\norm{\wh{\Sigma}_{xy}-\wh{\Sigma}_x(\lam\theta\eta\tp)\wh{\Sigma}_y}_\infty\\
&\leq \norm{\wh{\Sigma}_{xy}-\Sigma_{xy}}_\infty+\norm{ \Sigma_{x}(\lam\theta\eta\tp)\Sigma_{y} - \wh{\Sigma}_x(\lam\theta\eta\tp)\wh{\Sigma}_y  }_\infty
\eeqal
For the first part, we easily derive the bound
\beq
\norm{\wh{\Sigma}_{xy}-\Sigma_{xy}}_\infty\lesssim \sqrt{\frac{\log pq}{n}}
\eeq
For the second part, we first divide it into three parts
\beqal
\norm{ \Sigma_{x}(\lam\theta\eta\tp)\Sigma_{y} - \wh{\Sigma}_x(\lam\theta\eta\tp)\wh{\Sigma}_y  }_\infty&\lesssim \underbrace{\norm{\bra{\wh{\Sigma}_x-\Sigma_x}\theta\cdot \bra{\Sigma_{y}\eta}\tp}_\infty}_{(\text{\rom{1}})} +\underbrace{\norm{\bra{\Sigma_{x}\theta }\cdot \left[ \bra{\wh{\Sigma}_y-\Sigma_y}\eta\right]\tp}}_{(\text{\rom{2}})}\\
&+\underbrace{\norm{ \bra{\wh{\Sigma}_x-\Sigma_x}\theta \cdot \left[\bra{\wh{\Sigma}_y-\Sigma_{y}}\eta\right]\tp }_\infty}_{(\text{\rom{3}})}
\eeqal
We will only analyze \rom{1}, the rest is the same
\beqal
(\text{\rom{1}})&=\norm{\bra{\wh{\Sigma}_x-\Sigma_x}\theta}_\infty\cdot\norm{\Sigma_{y}\eta}_\infty\\
&\leq \max_{j}\ndone\sum_{i=1}^n\bra{X_{ij}X_i\tp \theta - \E X_{ij}X_i\tp \theta}\cdot \bra{\norm{\Sigma_{y}\srt \eta}_{op}\cdot \norm{\Sigma_{y}\srt \eta}}\\
&\lesssim \sqrt{M\frac{\log p}{n}}
\eeqal
The first inequality is derived because $\norm{Ax}_\infty\leq \norm{Ax}\leq \norm{A}_{op}\norm{x}$.
Now observe that
\beq
\log pq=\log p+\log q\asymp\log (p+q)
\eeq
Thus, we could choose $\rho = C\sqrt{\frac{\log (p+q)}{n}}$. Then w.h.p. we counclude that
\beq
\norm{\Delta}_F^2\lesssim \frac{s_1s_2\log(p+q)}{n\lam^2}
\eeq
Now note that
\beq
\norm{\wh{A}-A}_F^2\lesssim\norm{\Delta}_F^2+\norm{\wt{A}-A}_F^2
\eeq
And the second term in RHS, recall the definition~\ref{eq:Atilde}, we have
\beqal
\norm{\wt{A}-A}_F&\leq \abs{\frac{1}{\sqrt{\theta\tp\wh{\Sigma}_x}\theta\cdot \sqrt{\eta\tp\wh{\Sigma}_y}\eta}-1}\norm{\theta\eta\tp}_F\\
&\lesssim \frac{1}{\sqrt{n}}
\eeqal
So we conclude with the following theorem.
\begin{theorem}\label{thm:thm1}
Assume that $\frac{s_1s_2\log \bra{p+q}}{n\lam^2}$ is sufficiently small, and $\wh{A}$ is one of the solution of the convex relaxation program~\ref{eq:cvxrelax}, then w.h.p.
\beq
\norm{\wh{A}-\theta\eta\tp}_F^2\lesssim \frac{s_1s_2\log\bra{p+q}}{n\lam^2}
\eeq
\end{theorem}
Now let $\wh{\theta}$ be the leading left singular vector of $\wh{A}$, which is also the leading eigenvector of $\wh{A}\wh{A}\tp$. We first define the projection operator
\[
P_{v}=\frac{vv\tp}{\norm{v}^2}
\]
Then we see from Sine-Theta theorem
\beq
\norm{P_{\wh{\theta}}-P_\theta}_F^2\lesssim \norm{\wh{A}\wh{A}\tp-AA\tp}_F^2
\eeq
Note that
\beqal
\norm{\wh{A}\wh{A}\tp-AA\tp}_F&\leq \norm{\bra{\wh{A}-A}\cdot A\tp}_F+\norm{A\bra{\wh{A}-A}\tp}_F+\norm{\bra{\wh{A}-A}\cdot \bra{\wh{A}-A}\tp}_F\\
&\leq 2\norm{A}_{op}\cdot \norm{\wh{A}-A}_F+\norm{\wh{A}-A}_F^2\\
&\lesssim \norm{\wh{A}-A}_F
\eeqal
Thus
\beq\label{eq:init}
\norm{P_{\wh{\theta}}-P_\theta}_F^2\lesssim\frac{s_1s_2\log\bra{p+q}}{n\lam^2}
\eeq
\section{Improvements}
Let's not be satisfied with the error rate in~\ref{thm:thm1} and try to do some improvements as in~\cite{Ma_2013}. We first do the regression characterization.
\subsection{Regression}
Suppose we already know the true $\eta$, and we would like to do regression of $\eta\tp Y$ w.r.t. $X$, where
\beq
\Cov\begin{pmatrix}
X\\Y
\end{pmatrix}=
\begin{pmatrix}
\Sigma_{x}&\Sigma_{xy}\\
\Sigma_{yx}&\Sigma_{y}\\
\end{pmatrix}
\eeq
So we first investigate $\E \bra{\eta\tp Y-\beta\tp X}^2$
\beqal
\E \bra{\eta\tp Y-\beta\tp X}^2&=\E\left[ \bra{\eta\tp Y}^2 +\bra{\beta\tp X}^2 - 2\bra{\eta\tp Y}\bra{\beta\tp X} \right]\\
&=\eta\tp\Sigma_{y}\eta+\beta\tp\Sigma_{x}\beta-2\beta\tp \Sigma_{xy}\eta\\
&=\eta\tp\Sigma_{y}\eta+\beta\tp\Sigma_{x}\beta-2\lam \beta\tp \Sigma_{x}\theta\bra{\eta\tp \Sigma_{y}\eta}\\
\eeqal
We want to minimize this term, so we let the gradient of it equal to zero
\beq
\nabla_{\beta}\E \bra{\eta\tp Y-\beta\tp X}^2=2\Sigma_{x}\beta-2\lam\Sigma_{x}\theta=0
\eeq
And the solution to this minimization problem is
\beq
\beta^*=\lam \theta
\eeq
And if instead of knowing the true parameter $\eta$, we know $\bar{\eta}$, then the solution of the regression problem is
\beq\label{eq:regression}
\beta^*=\lam\bra{\eta\tp\Sigma_{y}\bar{\eta}}\theta
\eeq
So note that as long as the term $\eta\tp\Sigma_{y}\bar{\eta}$ is bounded away from zero, then we could normalize the vector and recover the true parameter $\theta$. This also means that the initialization of $\bar{\eta}$ does matter because in high dimensional case, there is large chance that $\Sigma_{y}\srt \eta$ and $\Sigma_{y}\srt\bar{\eta}$ are orthogonal, if we simply choose $\bar{\eta}$ randomly. But from the lecture in sparse PCA, we could use the leading (left or right) singular vector of the convex relaxation solver and then do the following iterative regression algorithm.
\begin{algorithm}[H]
\caption{Iterative Regression}
\begin{algorithmic}[1]
\Function{IterativeReg}{$X,Y,\rho_1,\rho_2,\bar{\eta}^{(0)},iter=1$}
\For{$t\in [iter]$}
\Let{$\theta^{(t)}$}{$\arg\min_{\theta\in\R^p}\ndone\sum\bra{Y_i\tp\bar{\eta}^{(t-1)}-X_i\tp\theta}^2+\rho_1\norm{\theta}_1$}
\Let{$\bar{\theta}^{(t)}$}{$\frac{\theta^{(t)}}{\sqrt{\theta^{(t)\top}} \wh{\Sigma}_x\theta^{(t)}}$}
\Let{$\eta^{(t)}$}{$\arg\min_{\eta\in\R^q}\ndone\sum\bra{X_i\tp\bar{\theta}^{(t)}- Y_i\tp{\eta}}^2+\rho_2\norm{\eta}_1$}
\Let{$\bar{\eta}^{(t)}$}{$\frac{\eta^{(t)}}{\sqrt{\eta^{(t)\top}} \wh{\Sigma}_y\eta^{(t)}}$}
\EndFor
\State \Return{$\bar{\theta}^{(t)},\bar{\eta}^{(t)}$}
\EndFunction
\end{algorithmic}
\end{algorithm}
And by the following analysis of the algorithm, we are going to see that we only need one iteration to achieve the minimax rate.
\subsection{Analysis of Iterative Regression}
Suppose we use the convex relaxation to initialize, i.e., we set $\eta^{(0)}$ as the leading right singular vector of $\wh{A}$ in~\ref{eq:cvxrelax}. And suppose $\wh{\theta}$ is the following lasso solver
\beq
\wh{\theta}=\arg\min_{\theta\in\R^p}\ndone\sum\bra{Y_i\tp \eta\idx{0}-X_i\tp\theta}^2+\rho\norm{\theta}_1
\eeq
Due to~\ref{eq:regression}, the regression is carried on towards
\beq\label{eq:thetatilde}
\wt{\theta}=\lam\bra{\eta\tp \Sigma_{y}\eta\idx{0}}\theta
\eeq
So we now write out the basic inequality
\beq
\ndone \sum\bra{ Y_i\tp\eta\idx{0}-X_i\tp \wh{\theta} }^2+\rho\norm{\wh{\theta}}_1\leq \ndone \sum\bra{ Y_i\tp\eta\idx{0}-X_i\tp \wt{\theta} }^2+\rho\norm{\wt{\theta}}_1
\eeq
Rearranging the order gives
\beqal
\ndone\sum\bra{X_i\tp\wh{\theta}-X_i\wt{\theta}}^2&\leq 2\abs{\ndone \sum\bra{Y_i\tp \eta\idx{0}-X_i\tp\wt{\theta} }\cdot \bra{X_i\tp \wh{\theta}-X_i\tp\wt{\theta} } }+\rho\bra{\norm{\wt{\theta}}_1-\norm{\wh{\theta}}_1 }\\
&=2\abs{\bra{\wh{\theta}-\wt{\theta}}\tp\bra{\wh{\Sigma}_{xy}\eta\idx{0}-\wh{\Sigma}_{x}\wt{\theta} }  }+\rho\bra{\norm{\wt{\theta}}_1-\norm{\wh{\theta}}_1 }\\
\eeqal
Let
\beq
\Delta = \wh{\theta}-\wt{\theta}
\eeq
and using Holder's inequality, we further deduce that
\beq
\Delta\tp \wh{\Sigma}_x\Delta\leq 2\norm{\wh{\Sigma}_{xy}\eta\idx{0}-\wh{\Sigma}_{x}\wt{\theta}}\cdot \norm{\Delta}_1+\rho\norm{\wt{\theta}}_1-\rho\norm{\wt{\theta}+\Delta}_1
\eeq
Doing those tricks as we do in the proof of Lasso, we could deduce that if 
\beq\label{eq:assump}
\rho \geq 4\norm{\wh{\Sigma}_{xy}\eta\idx{0}-\wh{\Sigma}_x\wt{\theta}}_\infty
\eeq
then w.h.p.
\mtline{
\Delta\tp \wh{\Sigma}_x\Delta&\leq\frac{3}{2}\rho\norm{\Delta_{S_1}}_1\\
\norm{\Delta_{S_1\com}}_1&\leq 3\norm{\Delta_{S_1}}_1\\
\Delta\tp \wh{\Sigma}_x\Delta&\gtrsim \norm{\Delta}^2\quad\text{(peeling technique)}\\
\norm{\Delta}^2&\lesssim s_1\rho^2
}
So now we only need to specify $\rho$, such that the assumption~\ref{eq:assump} holds. We analyze $\norm{\wh{\Sigma}_{xy}\eta\idx{0}-\wh{\Sigma}_x\wt{\theta}}_\infty$.
\beq\label{eq:rhoanal}
\norm{\wh{\Sigma}_{xy}\eta\idx{0}-\wh{\Sigma}_x\wt{\theta}}_\infty\leq \norm{\bra{\wh{\Sigma}_{xy}-\Sigma_{xy}}\eta\idx{0}}_\infty + \norm{{\Sigma}_{xy}\eta\idx{0}-{\Sigma}_x\wt{\theta}}_\infty+
\norm{\bra{\wh{\Sigma}_x-\Sigma_x}\wt{\theta}}_\infty
\eeq
Recall the definition of $\wt{\theta}$~\ref{eq:thetatilde}, we know
\beq
{\Sigma}_{xy}\eta\idx{0}-{\Sigma}_x\wt{\theta}=\Sigma_{x}\lam\theta\eta\tp\Sigma_{y}\eta\idx{0}-\Sigma_{x}\wt{\theta}=0
\eeq
So the second term of~\ref{eq:rhoanal} is eliminated. Note that if use the same data of $X$ and $Y$ all along, then $\eta\idx{0}$ is dependent on $X$ and $Y$ and then the error term in~\ref{eq:rhoanal} is hard to analyze. But we could actually use the so-called sample-splitting method, which means we separate data into halves, use the first half to do the initialization and get $\eta\idx{0}$ and the rest to do iterative regression. In that case, $\eta\idx{0}$ is independent of the data in the iterative regression process and thus, the first and third term of~\ref{eq:rhoanal} is easy to analyze and we could deduce
\beq
\norm{\wh{\Sigma}_{xy}\eta\idx{0}-\wh{\Sigma}_x\wt{\theta}}_\infty\lesssim \sqrt{\frac{\log p}{n}}
\eeq 
So if we choose $\rho = C\sqrt{\frac{\log p}{n}}$, then
\beq
\norm{\wh{\theta}-\lam\bra{\eta\tp\Sigma_{y}\eta\idx{0}}\theta}^2\lesssim \frac{s_1\log p}{n}
\eeq
But this is not what we want. We first note that
\beqal
\norm{\frac{x}{\norm{x}}-\frac{y}{\norm{y}}}&\leq \norm{\frac{x}{\norm{x}}-\frac{x}{\norm{y}}}+\norm{\frac{x}{\norm{y}}-\frac{y}{\norm{y}}}\\
&\leq \abs{\frac{1}{\norm{x}}-\frac{1}{\norm{y}}}\norm{x}+\frac{\norm{x-y}}{\norm{y}}\\
&=\frac{\norm{x}-\norm{y}+\norm{x-y}}{\norm{y}}\\
&\leq 2\frac{\norm{x-y}}{\norm{y}}
\eeqal
So we see that the error of two normalized vectors could be bounded using their original error. Now since $\wt{\theta}$ and $\theta$ are on the same direction, we have
\beqal
\min\bra{\norm{ \frac{\wh{\theta}}{\norm{\wh{\theta}}} - \frac{{\theta}}{\norm{{\theta}}} }^2, \norm{ \frac{\wh{\theta}}{\norm{\wh{\theta}}} + \frac{{\theta}}{\norm{{\theta}}} }^2}
&=\min\bra{\norm{ \frac{\wh{\theta}}{\norm{\wh{\theta}}} - \frac{\wt{\theta}}{\norm{\wt{\theta}}} }^2, \norm{ \frac{\wh{\theta}}{\norm{\wh{\theta}}} + \frac{\wt{\theta}}{\norm{\wt{\theta}}} }^2}\\
&\lesssim \frac{s_1\log p}{n\lam^2}\cdot \frac{1}{\bra{\eta\tp\Sigma_{y}\eta\idx{0}}^2}
\eeqal
We should also note that for unit vectors $u$ and $v$
\beq
\min\bra{\norm{u-v}^2, \norm{u+v}^2}=2\bra{1-\abs{u\tp v}}
\eeq
also
\beq
\norm{uu\tp-vv\tp}_F^2=2\bra{1-\abs{u\tp v}^2}=\min\bra{\norm{u-v}^2, \norm{u+v}^2}\bra{1+\abs{u\tp v}}
\eeq
Thus
\beq
\norm{uu\tp-vv\tp}_F^2\asymp \norm{u-v}^2
\eeq
So we also proved that
\beq
\norm{P_{\wh{\theta}} -\P_{\theta} }_F^2\lesssim \frac{s_1\log p}{n\lam^2}\cdot \frac{1}{\bra{\eta\tp\Sigma_{y}\eta\idx{0}}^2}
\eeq
The rest is to verify that $\eta\tp\Sigma_{y}\eta\idx{0}$ is indeed bounded away from zero if we used convex relaxation initialization. As we have already proved in~\ref{eq:init}
\beq
\norm{P_{\eta\idx{0}}-P_{\eta}}_F^2\lesssim \frac{s_1s_2\log\bra{p+q}}{n\lam^2}
\eeq
The LHS is also equal to
\beq
\norm{P_{\eta\idx{0}}-P_{\eta}}_F^2=2\bra{1-\abs{\eta^{(0)\top} \eta}^2}
\eeq
and the RHS is small because we assume that $\frac{s_1s_2\log\bra{p+q}}{n}$ is sufficiently small and $\lam$ is a constant. Now also note that the spectrum of $\Sigma_{y}$ is bounded below by $M\inv$, thus
\beq
\abs{\eta\tp \Sigma\eta\idx{0}}\geq \lam_{\min}^{\Sigma_{y}}\norm{\eta\tp\eta\idx{0}}\geq M\inv \norm{\eta\tp\eta\idx{0}}
\eeq
So now we could conclude with the following theorem
\begin{theorem}
Assume that $\frac{s_1s_2\log\bra{p+q}}{n\lam^2}$ is sufficiently small, then by using the two-stage algorithm with sample splitting, i.e., to use the first half of data to do the convex relaxation and initialize the setting, and use the rest to do the iterative regression (with one iteration), we achieve the minimax rate
\beq
\norm{P_{\wh{\theta}} -\P_{\theta} }_F^2\lesssim \frac{s_1\log p}{n\lam^2}
\eeq
\end{theorem}




















\bibliographystyle{apalike}
\bibliography{SparseCCA.bib}	
%%%%%%%%%%% end of doc
\end{document}