\documentclass[12pt]{article}
\usepackage[english]{babel}
\usepackage[utf8x]{inputenc}
\usepackage[T1]{fontenc}
\usepackage{scribe}
\usepackage{listings}

\usepackage{hyperref}
\usepackage{cleveref}
%\usepackage[style=authoryear]{biblatex}
\renewcommand{\Pr}{{\bf Pr}}
\newcommand{\Prx}{\mathop{\bf Pr\/}}
\newcommand{\E}{{\bf E}}
\newcommand{\Ex}{\mathop{\bf E\/}}
\newcommand{\Var}{{\bf Var}}
\newcommand{\Varx}{\mathop{\bf Var\/}}
\newcommand{\Cov}{{\bf Cov}}
\newcommand{\Cor}{{\bf Corr}}
\newcommand{\Covx}{\mathop{\bf Cov\/}}
\newcommand{\rank}{\text{rank}}

% shortcuts for symbol names that are too long to type
\newcommand{\eps}{\epsilon}
\newcommand{\lam}{\lambda}
\renewcommand{\l}{\ell}
\newcommand{\la}{\langle}
\newcommand{\ra}{\rangle}
%\newcommand{\wh}{\widehat}

% "blackboard-fonted" letters for the reals, naturals etc.
\newcommand{\R}{\mathbb R}
\newcommand{\N}{\mathbb N}
\newcommand{\Z}{\mathbb Z}
\newcommand{\F}{\mathbb F}
\newcommand{\Q}{\mathbb Q}
%\newcommand{\C}{\mathbb C}
\DeclareMathOperator{\Tr}{Tr}

% operators that should be typeset in Roman font
\newcommand{\poly}{\mathrm{poly}}
\newcommand{\polylog}{\mathrm{polylog}}
\newcommand{\sgn}{\mathrm{sgn}}
\newcommand{\avg}{\mathop{\mathrm{avg}}}
\newcommand{\val}{{\mathrm{val}}}

% complexity classes
\renewcommand{\P}{\mathrm{P}}
\newcommand{\NP}{\mathrm{NP}}
\newcommand{\BPP}{\mathrm{BPP}}
\newcommand{\DTIME}{\mathrm{DTIME}}
\newcommand{\ZPTIME}{\mathrm{ZPTIME}}
\newcommand{\BPTIME}{\mathrm{BPTIME}}
\newcommand{\NTIME}{\mathrm{NTIME}}

% values associated to optimization algorithm instances
\newcommand{\Opt}{{\mathsf{Opt}}}
\newcommand{\Alg}{{\mathsf{Alg}}}
\newcommand{\Lp}{{\mathsf{Lp}}}
\newcommand{\Sdp}{{\mathsf{Sdp}}}
\newcommand{\Exp}{{\mathsf{Exp}}}

% if you think the sum and product signs are too big in your math mode; x convention
% as in the probability operators
\newcommand{\littlesum}{{\textstyle \sum}}
\newcommand{\littlesumx}{\mathop{{\textstyle \sum}}}
\newcommand{\littleprod}{{\textstyle \prod}}
\newcommand{\littleprodx}{\mathop{{\textstyle \prod}}}

% horizontal line across the page
\newcommand{\horz}{
\vspace{-.4in}
\begin{center}
\begin{tabular}{p{\textwidth}}\\
\hline
\end{tabular}
\end{center}
}

% calligraphic letters
\newcommand{\calA}{{\cal A}}
\newcommand{\calB}{{\cal B}}
\newcommand{\calC}{{\cal C}}
\newcommand{\calD}{{\cal D}}
\newcommand{\calE}{{\cal E}}
\newcommand{\indep}{\rotatebox[origin=c]{90}{$\models$}}
\newcommand{\calF}{{\cal F}}
\newcommand{\calG}{{\cal G}}
\newcommand{\calH}{{\cal H}}
\newcommand{\calI}{{\cal I}}
\newcommand{\calJ}{{\cal J}}
\newcommand{\calK}{{\cal K}}
\newcommand{\calL}{{\cal L}}
\newcommand{\calM}{{\cal M}}
\newcommand{\calN}{{\cal N}}
\newcommand{\calO}{{\cal O}}
\newcommand{\calP}{{\cal P}}
\newcommand{\calQ}{{\cal Q}}
\newcommand{\calR}{{\cal R}}
\newcommand{\calS}{{\cal S}}
\newcommand{\calT}{{\cal T}}
\newcommand{\calU}{{\cal U}}
\newcommand{\calV}{{\cal V}}
\newcommand{\calW}{{\cal W}}
\newcommand{\calX}{{\cal X}}
\newcommand{\calY}{{\cal Y}}
\newcommand{\calZ}{{\cal Z}}
\newcommand{\com}{^\complement}
\newcommand{\inv}{^{-1}}
\newcommand{\tp}{^\top}
% bold letters (useful for random variables)
\renewcommand{\a}{{\boldsymbol a}}
\renewcommand{\b}{{\boldsymbol b}}
\renewcommand{\c}{{\boldsymbol c}}
\renewcommand{\d}{{\boldsymbol d}}
\newcommand{\e}{{\boldsymbol e}}
\newcommand{\f}{{\boldsymbol f}}
\newcommand{\g}{{\boldsymbol g}}
\newcommand{\h}{{\boldsymbol h}}
\renewcommand{\i}{{\boldsymbol i}}
\renewcommand{\j}{{\boldsymbol j}}
\renewcommand{\k}{{\boldsymbol k}}
\newcommand{\m}{{\boldsymbol m}}
\newcommand{\n}{{\boldsymbol n}}
\renewcommand{\o}{{\boldsymbol o}}
\newcommand{\p}{{\boldsymbol p}}
\newcommand{\q}{{\boldsymbol q}}
\renewcommand{\r}{{\boldsymbol r}}
\newcommand{\s}{{\boldsymbol s}}
\renewcommand{\t}{{\boldsymbol t}}
\renewcommand{\u}{{\boldsymbol u}}
\renewcommand{\v}{{\boldsymbol v}}
\newcommand{\w}{{\boldsymbol w}}
\newcommand{\x}{{\boldsymbol x}}
\newcommand{\y}{{\boldsymbol y}}
\newcommand{\z}{{\boldsymbol z}}
\newcommand{\A}{{\boldsymbol A}}
\newcommand{\B}{{\boldsymbol B}}
\newcommand{\D}{{\boldsymbol D}}
%\newcommand{\G}{{\boldsymbol G}}
\renewcommand{\H}{{\boldsymbol H}}
\newcommand{\I}{{\boldsymbol I}}
\newcommand{\J}{{\boldsymbol J}}
\newcommand{\K}{{\boldsymbol K}}
\renewcommand{\L}{{\boldsymbol L}}
\newcommand{\M}{{\boldsymbol M}}
\renewcommand{\O}{{\boldsymbol O}}
\renewcommand{\S}{{\boldsymbol S}}
\newcommand{\T}{{\boldsymbol T}}
%\newcommand{\U}{{\boldsymbol U}}
\newcommand{\V}{{\boldsymbol V}}
\newcommand{\W}{{\boldsymbol W}}
\newcommand{\X}{{\boldsymbol X}}
\newcommand{\Y}{{\boldsymbol Y}}
\newcommand{\bra}[1]{\left(#1\right)}
\newcommand{\abs}[1]{\left|#1\right|}
\newcommand{\innprod}[1]{\left\langle#1\right\rangle}
\newcommand{\norm}[1]{\left\|#1\right\|}
\newcommand{\wh}[1]{\widehat{#1}}
\newcommand{\wt}[1]{\widetilde{#1}}
\def\beq{\begin{equation}}
\def\eeq{\end{equation}}
\def\bal{\begin{aligned}}
\def\eal{\end{aligned}}
\def\beqal{\begin{equation}\begin{aligned}}
\def\eeqal{\end{aligned}\end{equation}}
\newcommand{\brkt}[1]{\left\{#1\right\}}
\newcommand{\mtline}[1]{\beq
\left\{
\bal
#1
\eal
\right.
\eeq}


% useful for Fourier analysis
\newcommand{\bits}{\{-1,1\}}
\newcommand{\bitsn}{\{-1,1\}^n}
\newcommand{\bn}{\bitsn}
\newcommand{\isafunc}{{: \bitsn \rightarrow \bits}}
\newcommand{\fisafunc}{{f : \bitsn \rightarrow \bits}}
\newcommand{\ndone}{\frac{1}{n}}

\makeatletter
\newcommand{\distas}[1]{\mathbin{\overset{#1}{\kern\z@\sim}}}%
\newsavebox{\mybox}\newsavebox{\mysim}
\newcommand{\distras}[1]{%
  \savebox{\mybox}{\hbox{\kern3pt$\scriptstyle#1$\kern3pt}}%
  \savebox{\mysim}{\hbox{$\sim$}}%
  \mathbin{\overset{#1}{\kern\z@\resizebox{\wd\mybox}{\ht\mysim}{$\sim$}}}%
}
\makeatother
\newcommand{\iid}{\distras{\text{i.i.d.}}}
\makeatletter
\newcommand*{\rom}[1]{\expandafter\@slowromancap\romannumeral #1@}
\makeatother

% if you want
\newcommand{\half}{{\textstyle \frac12}}

\newcommand{\myfig}[4]{\begin{figure}[h] \begin{center} \includegraphics[width=#1\textwidth]{#2} \caption{#3} \label{#4} \end{center} \end{figure}} 
\newcommand{\chara}{\mathbb{1}}

\Scribe{Xinze Li}
\Lecturer{Chao Gao}
\LectureNumber{07}
\LectureDate{Jan 29, 2020}
\LectureTitle{Precision Matrix Estimation}

\lstset{style=mystyle}



\begin{document}
	\MakeScribeTop
\section{Precision Matrix Estimator}
\subsection{Problem Formulation}
First suppose that $X_1,\cdots, X_n\iid \calN(0, \Sigma)$ and that $\Omega=\Sigma\inv$ is the precision matrix. We also assume the parameter space $\calG(p,s,m)$ as
\beq\label{eq:parameterspace}
\calG(p,s,m)=\left\{ 
\Omega:M\inv\leq\lam_{\min}(\Omega)\leq \lam_{\max}(\Omega)\leq M,\max_i\sum_{j=1}^p\chara(\Omega_{ij}\ne0)\leq s
\right\}
\eeq
Suppose $x_1\in \R^{p_1}$, $x_2\in\R^{p_2}$, and $p_1+p_2=p$
\beq
X=
\begin{pmatrix}
x_1\\
x_2
\end{pmatrix}
\sim 
\calN\bra{0, \begin{pmatrix}
\Sigma_{11}&\Sigma_{12}\\
\Sigma_{21}&\Sigma_{22}\\
\end{pmatrix}}
=
\calN\bra{0, \begin{pmatrix}
\Omega_{11}&\Omega_{12}\\
\Omega_{21}&\Omega_{22}\\
\end{pmatrix}^{-1}}
\eeq
And so we have the \href{https://stats.stackexchange.com/questions/30588/deriving-the-conditional-distributions-of-a-multivariate-normal-distribution}{conditional probability distribution} as follows
\beq
x_1|x_2\sim\calN\bra{-\Sigma_{12}\Sigma_{22}\inv x_2, \Sigma_{11}-\Sigma_{12}\Sigma_{22}\inv \Sigma_{21}}
\eeq
Now let $S=\Sigma_{11}-\Sigma_{12}\Sigma_{22}\inv \Sigma_{21}$ the Schur complement. From matrix theory, we know
\beq
\begin{pmatrix}
\Omega_{11}&\Omega_{12}\\
\Omega_{21}&\Omega_{22}\\
\end{pmatrix}=
\begin{pmatrix}
\Sigma_{11}&\Sigma_{12}\\
\Sigma_{21}&\Sigma_{22}\\
\end{pmatrix}\inv=
\begin{pmatrix}
S\inv&-S\inv \Sigma_{11}\Sigma_{22}\inv\\
*&*\\
\end{pmatrix}
\eeq
Thus we also have
\beq
x_1|x_2\sim\calN\bra{-\Omega_{11}^{-1}\Omega_{12}x_2, \Omega_{11}^{-1}}
\eeq
We denote $x_{-i}=x_{[p]\textbackslash i}\in\R^{p-1}$, and so
\beq\label{eq:condprob}
x_1|x_{-1}\sim \calN\bra{-\Omega_{11}\inv \Omega_{1,-1}x_{-1}, \Omega_{11}\inv}
\eeq
\subsection{Idea}
Now the idea is to apply Lasso to estimate each column of the precision matrix $\Omega$ (See the article~\cite{Meinshausen_2006}). Now recall that in Lasso, we assume that $y=X\beta+\sigma z$, and $z\sim \calN(0, I_n)$, $X\in \R^{n\times p}$. The goal is to recover both $\beta$ and $\theta$. Suppose $X=\bra{X_1,\cdots,X_n}\tp$, and that $X_1,\cdots,X_n\iid \calN(0,\Sigma)$. If we also assume that $\frac{s\log p}{n}$ is small and
\beq\label{eq:assump}
M\inv\leq \lam_{\min}(\Sigma)\leq\lam_{\max}(\Sigma)\leq M, \quad \sigma^2\leq M,\quad \beta\in\Theta(p,s)
\eeq
And that $\wh{\beta}$ is the Lasso estimator
\beq
\wh{\beta}=\arg\min_{\beta}\norm{y-X\beta}^2+\lam\norm{\beta}_1
\eeq
If we choose 
\beq
\lam=C\sqrt{n\log p}
\eeq
We then have
\beq
\left\{
\bal
\ndone\norm{X\bra{\wh{\beta}-\beta}}^2&\lesssim \frac{s\log p}{n\kappa^2}\\
\norm{\wh{\beta}-\beta}^2&\lesssim \frac{s\log p}{n\kappa^4}
\eal
\right.
\eeq
And if we have the above assumption, we could actually obtain $\kappa \gtrsim 1$. So we have
\beq\label{eq:lassoproperty}
\left\{
\bal
\ndone\norm{X\bra{\wh{\beta}-\beta}}^2&\lesssim \frac{s\log p}{n}\\
\norm{\wh{\beta}-\beta}^2&\lesssim \frac{s\log p}{n}
\eal
\right.
\eeq
Now using the cone condition
\beq
\norm{\wh{\beta}-\beta}_1=\norm{\Delta_S}_1+\norm{\Delta_{S\com}}_1\leq 4\norm{\Delta_S}_1\leq 4\sqrt{s}\norm{\Delta}\leq 4\sqrt{s}\norm{\wh{\beta}-\beta}
\eeq
And thus
\beq
\norm{\wh{\beta}-\beta}_1\lesssim s\sqrt{\frac{\log p}{n}}
\eeq
Now from \ref{eq:condprob}, we know that if we write
\beq\label{eq:set1}
x_1=\begin{pmatrix}
X_{11}\\
\vrule\\
X_{n1}
\end{pmatrix}\in \R^{n\times 1},\quad
x_{-1}=\begin{pmatrix}
X_{1,-1}\\ \vrule \\ X_{n,-1}
\end{pmatrix}\in \R^{n\times (p-1)}
\eeq
then we have
\beq
x_1=-\Omega_{11}\inv x_{-1}\tp \Omega_{-1,1}+\Omega_{11}^{-\frac{1}{2}}z, \quad z\sim \calN(0,I_{n}),\quad \Omega_{11}\in \R,\quad \Omega_{-1,1}\in\R^{(p-1)\times 1}
\eeq
And so we could do the following Lasso estimator
\mtline{
\wh{\beta}&=\arg\min_{\beta\in\R^{p-1}}\norm{x_1-x_{-1}\tp \beta}^2+\lam \norm{\beta}_1\\
\wh{\sigma}^2&=\ndone \norm{x_1-x_{-1}\tp \beta}^2}
Now we are ready to give the estimate of the first column of the precision matrix
\mtline{\label{eq:estimator}
\wh{\Omega}_{11}&=\frac{1}{\wh{\sigma}^2}\\
\wh{\Omega}_{-1,1}&=\frac{\wh{\beta}}{\wh{\sigma}^2}\\
}
\subsection{Analysis}
To do the analysis, we should first check that all the assumptions are satisfied. From \ref{eq:set1}, we know that $X_{1,-1},\cdots, X_{n,-1}\iid \calN(0,\Omega_{-1,-1}\inv)$, and here $\sigma^2=\Omega_{11}\inv$, $\beta=-\Omega_{11}\inv \Omega_{-1,1}$. Now since we assume in ~\ref{eq:parameterspace}, we know that 
\beq
M\inv \leq \lam_{\min}\bra{\Omega}\leq \lam_{\min}\bra{\Omega_{-1,-1}}\leq \lam_{\max}\bra{\Omega_{-1,-1}}\leq \lam_{\max}\bra{\Omega}\leq M
\eeq
But $\lam_{\max}\bra{\Omega_{-1,-1}\inv}=1/\lam_{\min}(\Omega_{-1,-1})$, and $\lam_{\min}\bra{\Omega_{-1,-1}\inv}=1/\lam_{\max}(\Omega_{-1,-1})$. Thus
\beq
M\inv \leq \lam_{\min}(\Omega_{-1,-1}\inv)\leq \lam_{\max}(\Omega_{-1,-1}\inv)\leq M
\eeq
By the same logic
\beq
M\inv\leq \sigma^2\leq M
\eeq
Also since $\sum \chara(\Omega_{1i}\ne 0)\leq s$, and $\Omega$ is symmetric, we have $\beta\in \Theta(p-1, s)$. Thus the assumptions ~\ref{eq:assump} are met. Now we are ready to do our analysis. First let's analyze $\abs{\wh{\sigma}^2-\sigma^2}$,
\beqal
\abs{\wh{\sigma}^2-\sigma^2}&=\abs{ \ndone\norm{X\bra{\beta-\wh{\beta}}+\sigma z} -\sigma^2 }\\
&\leq \ndone\norm{X(\beta-\wh{\beta})}^2+\sigma^2\abs{\frac{\norm{z}^2}{n}-1}+\frac{2\sigma}{n}\abs{z\tp X(\beta-\wh{\beta})}\\
&\lesssim \frac{s\log p}{n}+\sqrt{\frac{\log p}{n}}+\frac{2\sigma}{n}\norm{X\tp z}_\infty\cdot \norm{\beta-\wh{\beta}}_1\\
&\lesssim \frac{s\log p}{n}+\sqrt{\frac{\log p}{n}}+\frac{2\sigma}{n}\sqrt{n\log p}\cdot s\sqrt{\frac{\log p}{n}}\\
&\lesssim \frac{s\log p}{n}+\sqrt{\frac{\log p}{n}}
\eeqal
The second inequality is Lasso's property~\ref{eq:lassoproperty}, concentration of chi-square and Holder's inequality. The third inequality is union bound and Lasso's property~\ref{eq:lassoproperty}. Thus, w.h.p. the following holds
\beq\label{eq:anal1}
\abs{\wh{\sigma}^2-\sigma^2}\lesssim \frac{s\log p}{n}+\sqrt{\frac{\log p}{n}}
\eeq
And now we could analyze the error of the precision matrix estimator ~\ref{eq:estimator}
\beqal
\abs{\wh{\Omega}_{11}-\Omega_{11}}&=\abs{\frac{1}{\sigma^2}-\frac{1}{\wh{\sigma}^2}}\\
&=\frac{\abs{\wh{\sigma}^2-\sigma^2}}{\sigma^2\wh{\sigma}^2}\\
&\lesssim \abs{\wh{\sigma}^2-\sigma^2}\\&\lesssim\frac{s\log p}{n}+\sqrt{\frac{\log p}{n}}
\eeqal
The first inequality is because $\sigma^2\geq M\inv$. Also, $\wh{\sigma}^2$ and $\sigma^2$ are close due to~\ref{eq:anal1}. Thus, $\sigma^2\wh{\sigma}^2\gtrsim 1$.
\beqal
\norm{\wh{\Omega}_{-1,1}-\Omega_{-1,1}}_1&=\norm{\frac{\wh{\beta}}{\wh{\sigma}^2}-\frac{\beta}{\sigma^2}}\\
&\leq \norm{\frac{\wh{\beta}}{\wh{\sigma}^2}-\frac{\beta}{\wh{\sigma}^2}}_1+\norm{\frac{\beta}{\wh{\sigma}^2}-\frac{\beta}{\sigma^2}}_1\\
&=\frac{1}{\wh{\sigma}^2}\norm{\wh{\beta}-\beta}_1+\norm{\beta}_1\abs{\frac{1}{\wh{\sigma}^2}-\frac{1}{\sigma}^2}\\
&\lesssim s\sqrt{\frac{\log p}{n}}+\sqrt{s}\cdot \bra{\frac{s\log p}{n}+\sqrt{\frac{\log p}{n}}}\\
&\lesssim s\sqrt{\frac{\log p}{n}}
\eeqal
Note that we used the estimate $\norm{\beta}_1\lesssim \sqrt{s}$ in the second inequality. This is valid because
\beqal
\norm{\beta}_1&\leq \sqrt{s}\norm{\beta}\\&=\sqrt{s}\norm{\sigma^2\Omega_{-1,1}}\\
&\lesssim \sqrt{s}\sigma^2\norm{\Omega}\\&=\sqrt{s}\sigma^2\lam_{\max}\\
&\lesssim \sqrt{s}
\eeqal
Thus we have
\beq
\Pr\bra{ \norm{\wh{\Omega}_{*1}-\Omega_{*1}}_1 > Cs\sqrt{\log p /n} }\leq p^{-10}
\eeq
Using union bound, we have
\beq
\Pr\bra{ \max_{1\leq j\leq p}\norm{\wh{\Omega}_{*j}-\Omega_{*j}}_1 > Cs\sqrt{\log p /n} }\leq p^{-9}
\eeq
Thus the following holds w.h.p.
\beq
\norm{\wh{\Omega}-\Omega}_{op}^2\leq \norm{\wh{\Omega}-\Omega}_1^2\lesssim \frac{s^2\log p}{n}
\eeq
which is also the optimal rate (See~\cite{cai2012estimating})
\beq
\inf_{\wh{\Omega}}\sup_{\Omega\in\calG(p,s,m)}\E \norm{\wh{\Omega}-\Omega}_{op}^2\gtrsim \frac{s^2\log p}{n}
\eeq



















\bibliographystyle{apalike}
\bibliography{PrecisionEstimate.bib}	
%%%%%%%%%%% end of doc
\end{document}