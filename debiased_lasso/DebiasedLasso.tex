\documentclass[12pt]{article}
\usepackage[english]{babel}
\usepackage[utf8x]{inputenc}
\usepackage[T1]{fontenc}
\usepackage{scribe}
\usepackage{listings}

\usepackage{hyperref}
\usepackage{cleveref}
%\usepackage[style=authoryear]{biblatex}
\renewcommand{\Pr}{{\bf Pr}}
\newcommand{\Prx}{\mathop{\bf Pr\/}}
\newcommand{\E}{{\bf E}}
\newcommand{\Ex}{\mathop{\bf E\/}}
\newcommand{\Var}{{\bf Var}}
\newcommand{\Varx}{\mathop{\bf Var\/}}
\newcommand{\Cov}{{\bf Cov}}
\newcommand{\Covx}{\mathop{\bf Cov\/}}

% shortcuts for symbol names that are too long to type
\newcommand{\eps}{\epsilon}
\newcommand{\lam}{\lambda}
\renewcommand{\l}{\ell}
\newcommand{\la}{\langle}
\newcommand{\ra}{\rangle}
%\newcommand{\wh}{\widehat}

% "blackboard-fonted" letters for the reals, naturals etc.
\newcommand{\R}{\mathbb R}
\newcommand{\N}{\mathbb N}
\newcommand{\Z}{\mathbb Z}
\newcommand{\F}{\mathbb F}
\newcommand{\Q}{\mathbb Q}
%\newcommand{\C}{\mathbb C}
\DeclareMathOperator{\Tr}{Tr}

% operators that should be typeset in Roman font
\newcommand{\poly}{\mathrm{poly}}
\newcommand{\polylog}{\mathrm{polylog}}
\newcommand{\sgn}{\mathrm{sgn}}
\newcommand{\avg}{\mathop{\mathrm{avg}}}
\newcommand{\val}{{\mathrm{val}}}

% complexity classes
\renewcommand{\P}{\mathrm{P}}
\newcommand{\NP}{\mathrm{NP}}
\newcommand{\BPP}{\mathrm{BPP}}
\newcommand{\DTIME}{\mathrm{DTIME}}
\newcommand{\ZPTIME}{\mathrm{ZPTIME}}
\newcommand{\BPTIME}{\mathrm{BPTIME}}
\newcommand{\NTIME}{\mathrm{NTIME}}

% values associated to optimization algorithm instances
\newcommand{\Opt}{{\mathsf{Opt}}}
\newcommand{\Alg}{{\mathsf{Alg}}}
\newcommand{\Lp}{{\mathsf{Lp}}}
\newcommand{\Sdp}{{\mathsf{Sdp}}}
\newcommand{\Exp}{{\mathsf{Exp}}}

% if you think the sum and product signs are too big in your math mode; x convention
% as in the probability operators
\newcommand{\littlesum}{{\textstyle \sum}}
\newcommand{\littlesumx}{\mathop{{\textstyle \sum}}}
\newcommand{\littleprod}{{\textstyle \prod}}
\newcommand{\littleprodx}{\mathop{{\textstyle \prod}}}

% horizontal line across the page
\newcommand{\horz}{
\vspace{-.4in}
\begin{center}
\begin{tabular}{p{\textwidth}}\\
\hline
\end{tabular}
\end{center}
}

% calligraphic letters
\newcommand{\calA}{{\cal A}}
\newcommand{\calB}{{\cal B}}
\newcommand{\calC}{{\cal C}}
\newcommand{\calD}{{\cal D}}
\newcommand{\calE}{{\cal E}}
\newcommand{\indep}{\rotatebox[origin=c]{90}{$\models$}}
\newcommand{\calF}{{\cal F}}
\newcommand{\calG}{{\cal G}}
\newcommand{\calH}{{\cal H}}
\newcommand{\calI}{{\cal I}}
\newcommand{\calJ}{{\cal J}}
\newcommand{\calK}{{\cal K}}
\newcommand{\calL}{{\cal L}}
\newcommand{\calM}{{\cal M}}
\newcommand{\calN}{{\cal N}}
\newcommand{\calO}{{\cal O}}
\newcommand{\calP}{{\cal P}}
\newcommand{\calQ}{{\cal Q}}
\newcommand{\calR}{{\cal R}}
\newcommand{\calS}{{\cal S}}
\newcommand{\calT}{{\cal T}}
\newcommand{\calU}{{\cal U}}
\newcommand{\calV}{{\cal V}}
\newcommand{\calW}{{\cal W}}
\newcommand{\calX}{{\cal X}}
\newcommand{\calY}{{\cal Y}}
\newcommand{\calZ}{{\cal Z}}

% bold letters (useful for random variables)
\renewcommand{\a}{{\boldsymbol a}}
\renewcommand{\b}{{\boldsymbol b}}
\renewcommand{\c}{{\boldsymbol c}}
\renewcommand{\d}{{\boldsymbol d}}
\newcommand{\e}{{\boldsymbol e}}
\newcommand{\f}{{\boldsymbol f}}
\newcommand{\g}{{\boldsymbol g}}
\newcommand{\h}{{\boldsymbol h}}
\renewcommand{\i}{{\boldsymbol i}}
\renewcommand{\j}{{\boldsymbol j}}
\renewcommand{\k}{{\boldsymbol k}}
\newcommand{\m}{{\boldsymbol m}}
\newcommand{\n}{{\boldsymbol n}}
\renewcommand{\o}{{\boldsymbol o}}
\newcommand{\p}{{\boldsymbol p}}
\newcommand{\q}{{\boldsymbol q}}
\renewcommand{\r}{{\boldsymbol r}}
\newcommand{\s}{{\boldsymbol s}}
\renewcommand{\t}{{\boldsymbol t}}
\renewcommand{\u}{{\boldsymbol u}}
\renewcommand{\v}{{\boldsymbol v}}
\newcommand{\w}{{\boldsymbol w}}
\newcommand{\x}{{\boldsymbol x}}
\newcommand{\y}{{\boldsymbol y}}
\newcommand{\z}{{\boldsymbol z}}
\newcommand{\A}{{\boldsymbol A}}
\newcommand{\B}{{\boldsymbol B}}
\newcommand{\D}{{\boldsymbol D}}
%\newcommand{\G}{{\boldsymbol G}}
\renewcommand{\H}{{\boldsymbol H}}
\newcommand{\I}{{\boldsymbol I}}
\newcommand{\J}{{\boldsymbol J}}
\newcommand{\K}{{\boldsymbol K}}
\renewcommand{\L}{{\boldsymbol L}}
\newcommand{\M}{{\boldsymbol M}}
\renewcommand{\O}{{\boldsymbol O}}
\renewcommand{\S}{{\boldsymbol S}}
\newcommand{\T}{{\boldsymbol T}}
%\newcommand{\U}{{\boldsymbol U}}
\newcommand{\V}{{\boldsymbol V}}
\newcommand{\W}{{\boldsymbol W}}
\newcommand{\X}{{\boldsymbol X}}
\newcommand{\Y}{{\boldsymbol Y}}
\newcommand{\bra}[1]{\left(#1\right)}
\newcommand{\abs}[1]{\left|#1\right|}
\newcommand{\innprod}[1]{\left\langle#1\right\rangle}
\newcommand{\norm}[1]{\left\|#1\right\|}
\newcommand{\wh}[1]{\widehat{#1}}
\newcommand{\wt}[1]{\widetilde{#1}}
\def\beq{\begin{equation}}
\def\eeq{\end{equation}}
\def\bal{\begin{aligned}}
\def\eal{\end{aligned}}
\def\beqal{\begin{equation}\begin{aligned}}
\def\eeqal{\end{aligned}\end{equation}}


% useful for Fourier analysis
\newcommand{\bits}{\{-1,1\}}
\newcommand{\bitsn}{\{-1,1\}^n}
\newcommand{\bn}{\bitsn}
\newcommand{\isafunc}{{: \bitsn \rightarrow \bits}}
\newcommand{\fisafunc}{{f : \bitsn \rightarrow \bits}}

% if you want
\newcommand{\half}{{\textstyle \frac12}}

\newcommand{\myfig}[4]{\begin{figure}[h] \begin{center} \includegraphics[width=#1\textwidth]{#2} \caption{#3} \label{#4} \end{center} \end{figure}} 

\Scribe{Xinze Li}
\Lecturer{Chao Gao}
\LectureNumber{08}
\LectureDate{Feb 03, 2020}
\LectureTitle{Debiased Lasso}

\lstset{style=mystyle}



\begin{document}
	\MakeScribeTop

%#############################################################
%#############################################################
%#############################################################
%#############################################################

In previous lectures, we obtain estimators of precision matrix. Now it's time for us to do inference, which is a harder problem.
\section{Debiased Lasso}
First suppose that 
\beq
y=X\beta +z,\quad z\sim \calN(0,I_n)
\eeq
Also suppose that $X=\bra{X_1,\cdots,X_n}^\top$, $X_i\in\R^p$ and that $X_i$ are i.i.d. drawn from multivariate normal distribution $\calN(0, \Omega^{-1})$, where $\Omega = \Sigma$ is the precision matrix. And so we could write the ordinary least square estimator if $n>p$
\beq
\wh{\beta}_{OLS}=\bra{X^\top X}X^\top y\sim \calN\bra{\beta, \bra{X^\top X}^{-1}}
\eeq
Also suppose that $\wt{\beta}$ is the Lasso estimator
\beq
\wt{\beta} = \arg\max_\beta \bra{\norm{y-X\beta}^2-\lam\norm{\beta}_1}
\eeq
We know that $\E y = X\beta$ and thus $\E X^\top y=X^\top X\beta$. And so
\beq
\wt{\beta}-\wh{\beta}_{OLS}=\bra{X^\top X}^{-1}\bra{X^\top X\wt{\beta}-X^\top y}
\eeq
is approximately the bias. A tautology from this equation is
\beqal
\wh{\beta}&=\wt{\beta}+\bra{X^\top X}^{-1}\bra{X^\top y-X^\top X\wt{\beta}}\\
&=\wt{\beta}+\bra{\frac{1}{n}X^\top X}^{-1}\cdot \frac{1}{n}\bra{X^\top y-X^\top X\wt{\beta}}
\eeqal
So now our idea is to replace $\bra{\frac{1}{n}X^\top X}^{-1}$ by a matrix close to the precision matrix. Let's call this approximation matrix $M$, and now the in general we could write
\beq
\wh{\beta}=\wt{\beta}+M\cdot \frac{1}{n}\bra{X^\top y-X^\top X\wt{\beta}}
\eeq
This idea first appeard in the article~\cite{zhang2011confidence}. It is also rediscovered in article~\cite{javanmard2015debiasing} and \cite{van_de_Geer_2014}. The results in the latter two papers are a subset of \cite{zhang2011confidence}, which is hard to read. 
\subsection{Known Covariance}
If $\Sigma$ the covariance matrix is known, then a natural attempt is to plug in $M = \Omega=\Sigma^{-1}$. Suppose $\wh{\Sigma}=\frac{1}{n}X^\top X$ is the sample covariance matrix, we have
\beqal\label{eq:anal1}
\wh{\beta}&=\wt{\beta}+ \frac{1}{n}\Omega X^\top \bra{y-X\wt{\beta}}\\
&=\wt{\beta}+ \frac{1}{n}\Omega X^\top \bra{X\beta +z -X\wt{\beta}}\\
&= \wt{\beta}+ \frac{1}{n}\Omega X^\top X\beta + \frac{1}{n}\Omega X^\top z - \frac{1}{n}\Omega X^\top X\wt{\beta}\\
&=\beta + \frac{1}{n}\Omega X^\top z+\bra{I_p-\Omega \wh{\Sigma}}\bra{\wt{\beta}-\beta}
\eeqal
Rearranging the order gives
\beq\label{eq:anal2}
\sqrt{n}\bra{\wh{\beta}-\beta}=\frac{1}{\sqrt{n}}\Omega X^\top z+\sqrt{n}\bra{I_p-\Omega \wh{\Sigma}}\bra{\wt{\beta}-\beta}
\eeq
Since inference is applied on each component, let's analyze each component of this vector.
\beq\label{eq:anal3}
\sqrt{n}\bra{\wh{\beta}_j-\beta_j}=\frac{1}{\sqrt{n}}\Omega_j^\top X^\top z+\sqrt{n}\bra{e_j-\wh{\Sigma}\Omega_j}^{\top} \cdot \bra{\wt{\beta}-\beta}
\eeq
Clearly the first term of the RHS is Gaussian given X. 
\beq\label{eq:anal4}
\frac{1}{\sqrt{n}}\Omega_j X^\top z|X \sim \calN\bra{0, \Omega_j^\top \wh{\Sigma}\Omega_j} 
\eeq
And for the second term, we use Holder's inequality
\beqal\label{eq:anal}
\abs{\sqrt{n}\bra{e_j-  \wh{\Sigma}\Omega_j}^\top \bra{\wt{\beta}-\beta}}&\leq \sqrt{n}\norm{e_j-  \wh{\Sigma}\Omega_j}_\infty\cdot \norm{\wt{\beta}-\beta}_1
\eeqal
For the latter part of the RHS $\norm{\wt{\beta}-\beta}_1$, we know from properties of lasso estimator that w.h.p.
\beq\label{eq:second part}
\norm{\wt{\beta}-\beta}_1\lesssim s\sqrt{\frac{\log p}{n}}
\eeq
And for the former part of RHS, we have
\beqal\label{eq:first part}
&\norm{e_j-  \wh{\Sigma}\Omega_j}_\infty\\
=& \max_{1\leq l\leq p}\frac{1}{n}\abs{\sum_{i=1}^{n} \left[ \bra{\Omega_j^\top X_i}\bra{ e_l^\top X_i }- \E \bra{\Omega_j^\top X_i}\bra{ e_l^\top X_i }  \right] }\\
\leq &\sqrt{\frac{\log p}{n}}
\eeqal
Details see HW5 Problem2 solution. So combining equation~\ref{eq:first part} and ~\ref{eq:second part} and plugging the bound into ~\ref{eq:anal} gives the following upper bound
\beq
\abs{\sqrt{n}\bra{e_j-\Omega_j^\top \wh{\Sigma}}\bra{\wt{\beta}-\beta}}\lesssim s\frac{\log p}{\sqrt{n}}
\eeq
Thus, if $\frac{s\log p}{\sqrt{n}}\rightarrow 0 $ and $\Omega_j^\top \wh{\Sigma}\Omega_j\gtrsim 1$ not degenerate w.h.p. (to be proved), then
\beq
\frac{\sqrt{n}\bra{\wh{\beta}_j-\beta_j}}{\sqrt{\Omega_j^\top \wh{\Sigma}\Omega_j}}\leadsto \calN(0,1)
\eeq
\subsection{Unknown Covariance}
Following the process as in \cref{eq:anal1,eq:anal2,eq:anal3,eq:anal4,eq:anal}, suppose $M=\bra{m_1, m_2,\cdots, m_p}^\top$ and that $M$ depends only on $X$, we deduce that
\beqal[c]
\wh{\beta}=\beta + \frac{1}{n}M X^\top z+\bra{I_p-M \wh{\Sigma}}\bra{\wt{\beta}-\beta}\\
\sqrt{n}\bra{\wh{\beta}-\beta}=\frac{1}{\sqrt{n}}M X^\top z+\sqrt{n}\bra{I_p-M \wh{\Sigma}}\bra{\wt{\beta}-\beta}\\
\sqrt{n}\bra{\wh{\beta}_j-\beta_j}=\frac{1}{\sqrt{n}}m_j^\top X^\top z+\sqrt{n}\bra{e_j- \wh{\Sigma}m_j}^\top\bra{\wt{\beta}-\beta}\\
\frac{1}{\sqrt{n}}m_j^\top X^\top z|X \sim \calN\bra{0, m_j^\top \wh{\Sigma}m_j} \\
\abs{\sqrt{n}\bra{e_j-\wh{\Sigma}m_j}^\top\bra{\wt{\beta}-\beta}}\leq \sqrt{n}\norm{e_j-\wh{\Sigma}m_j}_\infty\cdot \norm{\wt{\beta}-\beta}_1
\eeqal
In order to obtain the normality, note that we should restrict $\norm{e_j-\wh{\Sigma}m_j}_\infty$ to a scale of $\sqrt{\frac{\log}{n}}$ such that $\abs{\sqrt{n}\bra{e_j- \wh{\Sigma}m_j}\bra{\wt{\beta}-\beta}}\lesssim s\frac{\log p}{\sqrt{n}}$ as before. In the same time, we would like the confidence interval be as short as possible. In order to do this, we introduce the following convex program (j is fixed)
\beqal\label{cvx}
\min_{m\in \R^p}& \quad m^\top \wh{\Sigma}m\\
s.t.&\quad \norm{e_j-\wh{\Sigma}m_j}_\infty\leq C \sqrt{\frac{\log p}{n}} 
\eeqal
Note that this problem is certainly feasible since we have just proved that $\Omega_j$ is a feasible point in ~\ref{eq:first part}. Let $m_j$ be the solution to~\ref{cvx}, then as we previously stated, if $\frac{s\log p}{\sqrt{n}}\rightarrow 0 $ and $m_j^\top \wh{\Sigma}m_j\gtrsim 1$ not degenerate w.h.p. (to be proved), then
\beq
\frac{\sqrt{n}\bra{\wh{\beta}_j-\beta_j}}{\sqrt{m_j^\top \wh{\Sigma}m_j}}\leadsto \calN(0,1)
\eeq
Now let's prove that $m_j^\top \wh{\Sigma}m_j$ is indeed bounded away from zero. Suppose $\mu = C
\sqrt{\frac{\log p}{n}}$. Since $\norm{e_j- \wh{\Sigma}m_j}_\infty\leq \mu $, we have
\beq
\abs{e_j^\top \wh{\Sigma}m_j-1}\leq \mu
\eeq
And so
\beqal
m^\top \wh{\Sigma}m&\geq m^\top \wh{\Sigma}m+ce_j^\top \wh{\Sigma}m-ce_j^\top \wh{\Sigma}m\\
&\geq m^\top \wh{\Sigma}m+c(1-\mu)-ce_j^\top \wh{\Sigma}m\\
\eeqal
Taking minimum on the both sides gives
\beqal
\min_{\text{m feasible}}m^\top \wh{\Sigma}m&\geq c(1-\mu)+\min_{m\in \R^p}\bra{ m^\top \wh{\Sigma}m-ce_j^\top \wh{\Sigma}m}\\
&\geq c(1-\mu)-\frac{c^2}{4}e_j^\top \wh{\Sigma}e_j\\
&\geq \frac{\bra{1-\mu}^2 }{\wh{\Sigma}_{jj}}
\eeqal
Now note that $\mu = C
\sqrt{\frac{\log p}{n}}$ is small. $\wh{\Sigma}_{jj}=\Sigma_{jj}-\bra{\wh{\Sigma}_{jj}-\Sigma_{jj}}$, $\Sigma_{jj}$ is bounded away from zero, and $\wh{\Sigma}_{jj}-\Sigma_{jj}$ is of order $O\bra{\frac{1}{\sqrt{n}}}$ goes to zero. We conclude with the following theorem
\begin{theorem}
If $\frac{s\log p}{\sqrt{n}}\leadsto 0$, then
\beq
\frac{\sqrt{n}\bra{\wh{\beta}_j-\beta_j}}{\sqrt{m_j^\top \wh{\Sigma}m_j}}\leadsto \calN(0,1)
\eeq
\end{theorem}
\subsection{Estimate Noise Level}
If $y=X\beta +\sigma z$, then we could use lasso residual (not debiased lasso) to estimate the noise level $\sigma$
\beqal
\wh{\sigma}^2&=\frac{1}{n}\norm{y-X\wt{\beta}}^2\\
&=\frac{1}{n}\norm{X\beta-X\wt{\beta}+\sigma z}^2\\
&=\frac{\sigma^2}{n}\norm{z}^2+\frac{1}{n}\norm{X\bra{\beta-\wt{\beta}}}^2+\frac{2\sigma}{n}z^\top X\bra{\beta-\wt{\beta}}
\eeqal
And from bound deduced in last lecture, we deduce that
\beq
\sqrt{n}\bra{\wh{\sigma}^2-\sigma^2}=\sigma^2\frac{1}{\sqrt{n}}\sum_{i=1}^n\bra{z_i^2-1}+O_p\bra{\frac{s\log p}{\sqrt{n}}}
\eeq
Note that $\sigma^2\frac{1}{\sqrt{n}}\sum_{i=1}^n\bra{z_i^2-1}\leadsto \calN\bra{0,2\sigma^4} $. Thus we have
\beq
\frac{\sqrt{n}\bra{\wh{\sigma}^2-\sigma^2}}{\sqrt{2}\sigma^2}\leadsto \calN(0,1)
\eeq
%%%%%%%%%%% If you don't have citations then comment the lines below:
%


\bibliographystyle{apalike}
\bibliography{DebiasedLasso.bib}	
%%%%%%%%%%% end of doc
\end{document}